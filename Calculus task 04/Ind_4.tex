 \documentclass[11pt]{report}

\usepackage[T2A]{fontenc}

\usepackage[utf8]{inputenc}

\usepackage[russian]{babel}

\usepackage{amsmath,amssymb}

\usepackage{graphicx}

\graphicspath{ {d:/HSE/OR/CW/CW5pict/} }

\begin{document}

\pagestyle{empty}

{\bf Индивидуальное задание.}

 Вариант N1

Построить в одной координатной плоскости графики функций $f(x) = $
    $x^{2}$, $g(x) = $
    $2 \cos{\left(3 x \right)}$ на 
    отрезке $\left[ 0, \  \pi\right]$, $f(x)$ зеленая 
    пунктирная линия, $g(x)$ фиолетовая пунктирная линия. 
    Отметки на горизонтальной оси от $0$ до $\pi$ с 
    шагом $\pi / 6$, отметки подписать формулами как в Примере 3.  
    По вертикальной оси отметки $-2$, 0, $2$.

Вариант N2

Построить в одной координатной плоскости графики функций $f(x) = $
    $0$, $g(x) = $
    $2 \sin{\left(2 x \right)}$ на 
    отрезке $\left[ 3 \pi / 2, \  5 \pi / 2\right]$, $f(x)$ черная 
    пунктирная линия, $g(x)$ зеленая линия из точек. 
    Отметки на горизонтальной оси от $3 \pi / 2$ до $5 \pi / 2$ с 
    шагом $\pi / 4$, отметки подписать формулами как в Примере 3.  
    По вертикальной оси отметки $-2$, 0, $2$.

Вариант N3

Построить в одной координатной плоскости графики функций $f(x) = $
    $x^{2} - 4$, $g(x) = $
    $4 \cos{\left(2 x \right)}$ на 
    отрезке $\left[ 0, \  \pi\right]$, $f(x)$ черная 
    пунктирная линия, $g(x)$ желтая пунктирная линия. 
    Отметки на горизонтальной оси от $0$ до $\pi$ с 
    шагом $\pi / 4$, отметки подписать формулами как в Примере 3.  
    По вертикальной оси отметки $-4$, 0, $4$.

Вариант N4

Построить в одной координатной плоскости графики функций $f(x) = $
    $2 x + 4$, $g(x) = $
    $4 \sin{\left(4 x \right)}$ на 
    отрезке $\left[ 3 \pi / 4, \  3 \pi / 2\right]$, $f(x)$ желтая 
    пунктирная линия, $g(x)$ зеленая пунктирная линия. 
    Отметки на горизонтальной оси от $3 \pi / 4$ до $3 \pi / 2$ с 
    шагом $\pi / 8$, отметки подписать формулами как в Примере 3.  
    По вертикальной оси отметки $-4$, 0, $4$.

Вариант N5

Построить в одной координатной плоскости графики функций $f(x) = $
    $4 x - 2$, $g(x) = $
    $2 \sin{\left(4 x \right)}$ на 
    отрезке $\left[ \pi / 4, \  \pi / 2\right]$, $f(x)$ фиолетовая 
    сплошная линия, $g(x)$ желтая линия из точек. 
    Отметки на горизонтальной оси от $\pi / 4$ до $\pi / 2$ с 
    шагом $\pi / 8$, отметки подписать формулами как в Примере 3.  
    По вертикальной оси отметки $-2$, 0, $2$.

Вариант N6

Построить в одной координатной плоскости графики функций $f(x) = $
    $- x^{2} - x$, $g(x) = $
    $2 \sin{\left(3 x \right)}$ на 
    отрезке $\left[ \pi, \  5 \pi / 3\right]$, $f(x)$ черная 
    сплошная линия, $g(x)$ зеленая пунктирная линия. 
    Отметки на горизонтальной оси от $\pi$ до $5 \pi / 3$ с 
    шагом $\pi / 6$, отметки подписать формулами как в Примере 3.  
    По вертикальной оси отметки $-2$, 0, $2$.

Вариант N7

Построить в одной координатной плоскости графики функций $f(x) = $
    $2 x - 3$, $g(x) = $
    $3 \sin{\left(2 x \right)}$ на 
    отрезке $\left[ 3 \pi / 2, \  2 \pi\right]$, $f(x)$ красная 
    линия из точек, $g(x)$ синяя линия из точек. 
    Отметки на горизонтальной оси от $3 \pi / 2$ до $2 \pi$ с 
    шагом $\pi / 4$, отметки подписать формулами как в Примере 3.  
    По вертикальной оси отметки $-3$, 0, $3$.

Вариант N8

Построить в одной координатной плоскости графики функций $f(x) = $
    $- x^{2} + x + 3$, $g(x) = $
    $3 \sin{\left(2 x \right)}$ на 
    отрезке $\left[ \pi / 2, \  2 \pi\right]$, $f(x)$ синяя 
    сплошная линия, $g(x)$ фиолетовая линия из точек. 
    Отметки на горизонтальной оси от $\pi / 2$ до $2 \pi$ с 
    шагом $\pi / 4$, отметки подписать формулами как в Примере 3.  
    По вертикальной оси отметки $-3$, 0, $3$.

Вариант N9

Построить в одной координатной плоскости графики функций $f(x) = $
    $- x^{2} + 2 x$, $g(x) = $
    $2 \sin{\left(3 x \right)}$ на 
    отрезке $\left[ \pi / 3, \  \pi\right]$, $f(x)$ желтая 
    пунктирная линия, $g(x)$ зеленая линия из точек. 
    Отметки на горизонтальной оси от $\pi / 3$ до $\pi$ с 
    шагом $\pi / 6$, отметки подписать формулами как в Примере 3.  
    По вертикальной оси отметки $-2$, 0, $2$.

Вариант N10

Построить в одной координатной плоскости графики функций $f(x) = $
    $3 x - 2$, $g(x) = $
    $2 \cos{\left(4 x \right)}$ на 
    отрезке $\left[ 0, \  \pi / 2\right]$, $f(x)$ желтая 
    сплошная линия, $g(x)$ синяя пунктирная линия. 
    Отметки на горизонтальной оси от $0$ до $\pi / 2$ с 
    шагом $\pi / 8$, отметки подписать формулами как в Примере 3.  
    По вертикальной оси отметки $-2$, 0, $2$.

Вариант N11

Построить в одной координатной плоскости графики функций $f(x) = $
    $0$, $g(x) = $
    $2 \cos{\left(2 x \right)}$ на 
    отрезке $\left[ 3 \pi / 2, \  3 \pi\right]$, $f(x)$ красная 
    сплошная линия, $g(x)$ зеленая пунктирная линия. 
    Отметки на горизонтальной оси от $3 \pi / 2$ до $3 \pi$ с 
    шагом $\pi / 4$, отметки подписать формулами как в Примере 3.  
    По вертикальной оси отметки $-2$, 0, $2$.

Вариант N12

Построить в одной координатной плоскости графики функций $f(x) = $
    $x + 2$, $g(x) = $
    $2 \sin{\left(4 x \right)}$ на 
    отрезке $\left[ \pi / 4, \  \pi\right]$, $f(x)$ фиолетовая 
    сплошная линия, $g(x)$ черная пунктирная линия. 
    Отметки на горизонтальной оси от $\pi / 4$ до $\pi$ с 
    шагом $\pi / 8$, отметки подписать формулами как в Примере 3.  
    По вертикальной оси отметки $-2$, 0, $2$.

Вариант N13

Построить в одной координатной плоскости графики функций $f(x) = $
    $x^{2} - x$, $g(x) = $
    $3 \cos{\left(3 x \right)}$ на 
    отрезке $\left[ \pi / 3, \  4 \pi / 3\right]$, $f(x)$ черная 
    сплошная линия, $g(x)$ зеленая пунктирная линия. 
    Отметки на горизонтальной оси от $\pi / 3$ до $4 \pi / 3$ с 
    шагом $\pi / 6$, отметки подписать формулами как в Примере 3.  
    По вертикальной оси отметки $-3$, 0, $3$.

Вариант N14

Построить в одной координатной плоскости графики функций $f(x) = $
    $4 x$, $g(x) = $
    $4 \cos{\left(4 x \right)}$ на 
    отрезке $\left[ \pi / 2, \  5 \pi / 4\right]$, $f(x)$ черная 
    сплошная линия, $g(x)$ зеленая пунктирная линия. 
    Отметки на горизонтальной оси от $\pi / 2$ до $5 \pi / 4$ с 
    шагом $\pi / 8$, отметки подписать формулами как в Примере 3.  
    По вертикальной оси отметки $-4$, 0, $4$.

Вариант N15

Построить в одной координатной плоскости графики функций $f(x) = $
    $- x^{2} - x + 2$, $g(x) = $
    $2 \cos{\left(2 x \right)}$ на 
    отрезке $\left[ \pi / 2, \  3 \pi / 2\right]$, $f(x)$ красная 
    пунктирная линия, $g(x)$ зеленая пунктирная линия. 
    Отметки на горизонтальной оси от $\pi / 2$ до $3 \pi / 2$ с 
    шагом $\pi / 4$, отметки подписать формулами как в Примере 3.  
    По вертикальной оси отметки $-2$, 0, $2$.

Вариант N16

Построить в одной координатной плоскости графики функций $f(x) = $
    $0$, $g(x) = $
    $2 \cos{\left(2 x \right)}$ на 
    отрезке $\left[ 3 \pi / 2, \  3 \pi\right]$, $f(x)$ красная 
    линия из точек, $g(x)$ синяя пунктирная линия. 
    Отметки на горизонтальной оси от $3 \pi / 2$ до $3 \pi$ с 
    шагом $\pi / 4$, отметки подписать формулами как в Примере 3.  
    По вертикальной оси отметки $-2$, 0, $2$.

Вариант N17

Построить в одной координатной плоскости графики функций $f(x) = $
    $- x^{2} + 4 x - 3$, $g(x) = $
    $3 \sin{\left(3 x \right)}$ на 
    отрезке $\left[ \pi / 3, \  2 \pi / 3\right]$, $f(x)$ красная 
    линия из точек, $g(x)$ зеленая пунктирная линия. 
    Отметки на горизонтальной оси от $\pi / 3$ до $2 \pi / 3$ с 
    шагом $\pi / 6$, отметки подписать формулами как в Примере 3.  
    По вертикальной оси отметки $-3$, 0, $3$.

Вариант N18

Построить в одной координатной плоскости графики функций $f(x) = $
    $- 2 x$, $g(x) = $
    $2 \sin{\left(2 x \right)}$ на 
    отрезке $\left[ 3 \pi / 2, \  5 \pi / 2\right]$, $f(x)$ красная 
    сплошная линия, $g(x)$ зеленая линия из точек. 
    Отметки на горизонтальной оси от $3 \pi / 2$ до $5 \pi / 2$ с 
    шагом $\pi / 4$, отметки подписать формулами как в Примере 3.  
    По вертикальной оси отметки $-2$, 0, $2$.

Вариант N19

Построить в одной координатной плоскости графики функций $f(x) = $
    $x^{2} + 3 x$, $g(x) = $
    $2 \cos{\left(3 x \right)}$ на 
    отрезке $\left[ 0, \  \pi\right]$, $f(x)$ синяя 
    сплошная линия, $g(x)$ черная пунктирная линия. 
    Отметки на горизонтальной оси от $0$ до $\pi$ с 
    шагом $\pi / 6$, отметки подписать формулами как в Примере 3.  
    По вертикальной оси отметки $-2$, 0, $2$.

Вариант N20

Построить в одной координатной плоскости графики функций $f(x) = $
    $x^{2} - 2 x$, $g(x) = $
    $4 \cos{\left(3 x \right)}$ на 
    отрезке $\left[ \pi, \  4 \pi / 3\right]$, $f(x)$ зеленая 
    сплошная линия, $g(x)$ фиолетовая пунктирная линия. 
    Отметки на горизонтальной оси от $\pi$ до $4 \pi / 3$ с 
    шагом $\pi / 6$, отметки подписать формулами как в Примере 3.  
    По вертикальной оси отметки $-4$, 0, $4$.

Вариант N21

Построить в одной координатной плоскости графики функций $f(x) = $
    $x^{2} - x + 2$, $g(x) = $
    $2 \cos{\left(2 x \right)}$ на 
    отрезке $\left[ 3 \pi / 2, \  5 \pi / 2\right]$, $f(x)$ синяя 
    сплошная линия, $g(x)$ фиолетовая линия из точек. 
    Отметки на горизонтальной оси от $3 \pi / 2$ до $5 \pi / 2$ с 
    шагом $\pi / 4$, отметки подписать формулами как в Примере 3.  
    По вертикальной оси отметки $-2$, 0, $2$.

Вариант N22

Построить в одной координатной плоскости графики функций $f(x) = $
    $3 x - 2$, $g(x) = $
    $2 \cos{\left(4 x \right)}$ на 
    отрезке $\left[ \pi / 2, \  \pi\right]$, $f(x)$ желтая 
    пунктирная линия, $g(x)$ синяя линия из точек. 
    Отметки на горизонтальной оси от $\pi / 2$ до $\pi$ с 
    шагом $\pi / 8$, отметки подписать формулами как в Примере 3.  
    По вертикальной оси отметки $-2$, 0, $2$.

Вариант N23

Построить в одной координатной плоскости графики функций $f(x) = $
    $2 x$, $g(x) = $
    $4 \cos{\left(2 x \right)}$ на 
    отрезке $\left[ \pi, \  3 \pi / 2\right]$, $f(x)$ черная 
    сплошная линия, $g(x)$ желтая линия из точек. 
    Отметки на горизонтальной оси от $\pi$ до $3 \pi / 2$ с 
    шагом $\pi / 4$, отметки подписать формулами как в Примере 3.  
    По вертикальной оси отметки $-4$, 0, $4$.

Вариант N24

Построить в одной координатной плоскости графики функций $f(x) = $
    $x^{2} - x$, $g(x) = $
    $3 \cos{\left(3 x \right)}$ на 
    отрезке $\left[ \pi / 3, \  4 \pi / 3\right]$, $f(x)$ черная 
    линия из точек, $g(x)$ зеленая пунктирная линия. 
    Отметки на горизонтальной оси от $\pi / 3$ до $4 \pi / 3$ с 
    шагом $\pi / 6$, отметки подписать формулами как в Примере 3.  
    По вертикальной оси отметки $-3$, 0, $3$.

Вариант N25

Построить в одной координатной плоскости графики функций $f(x) = $
    $2 x + 4$, $g(x) = $
    $4 \cos{\left(2 x \right)}$ на 
    отрезке $\left[ \pi / 2, \  3 \pi / 2\right]$, $f(x)$ желтая 
    пунктирная линия, $g(x)$ синяя пунктирная линия. 
    Отметки на горизонтальной оси от $\pi / 2$ до $3 \pi / 2$ с 
    шагом $\pi / 4$, отметки подписать формулами как в Примере 3.  
    По вертикальной оси отметки $-4$, 0, $4$.

Вариант N26

Построить в одной координатной плоскости графики функций $f(x) = $
    $4 x - 2$, $g(x) = $
    $2 \cos{\left(2 x \right)}$ на 
    отрезке $\left[ \pi, \  2 \pi\right]$, $f(x)$ фиолетовая 
    сплошная линия, $g(x)$ черная линия из точек. 
    Отметки на горизонтальной оси от $\pi$ до $2 \pi$ с 
    шагом $\pi / 4$, отметки подписать формулами как в Примере 3.  
    По вертикальной оси отметки $-2$, 0, $2$.

Вариант N27

Построить в одной координатной плоскости графики функций $f(x) = $
    $- x^{2} - 2 x + 2$, $g(x) = $
    $2 \sin{\left(2 x \right)}$ на 
    отрезке $\left[ \pi / 2, \  2 \pi\right]$, $f(x)$ фиолетовая 
    пунктирная линия, $g(x)$ черная линия из точек. 
    Отметки на горизонтальной оси от $\pi / 2$ до $2 \pi$ с 
    шагом $\pi / 4$, отметки подписать формулами как в Примере 3.  
    По вертикальной оси отметки $-2$, 0, $2$.

Вариант N28

Построить в одной координатной плоскости графики функций $f(x) = $
    $x$, $g(x) = $
    $2 \cos{\left(3 x \right)}$ на 
    отрезке $\left[ 2 \pi / 3, \  5 \pi / 3\right]$, $f(x)$ желтая 
    пунктирная линия, $g(x)$ зеленая пунктирная линия. 
    Отметки на горизонтальной оси от $2 \pi / 3$ до $5 \pi / 3$ с 
    шагом $\pi / 6$, отметки подписать формулами как в Примере 3.  
    По вертикальной оси отметки $-2$, 0, $2$.

Вариант N29

Построить в одной координатной плоскости графики функций $f(x) = $
    $- x^{2} - x - 2$, $g(x) = $
    $2 \sin{\left(2 x \right)}$ на 
    отрезке $\left[ \pi / 2, \  \pi\right]$, $f(x)$ желтая 
    сплошная линия, $g(x)$ синяя линия из точек. 
    Отметки на горизонтальной оси от $\pi / 2$ до $\pi$ с 
    шагом $\pi / 4$, отметки подписать формулами как в Примере 3.  
    По вертикальной оси отметки $-2$, 0, $2$.

Вариант N30

Построить в одной координатной плоскости графики функций $f(x) = $
    $- x^{2} - 2 x$, $g(x) = $
    $3 \cos{\left(3 x \right)}$ на 
    отрезке $\left[ 0, \  \pi\right]$, $f(x)$ черная 
    сплошная линия, $g(x)$ зеленая пунктирная линия. 
    Отметки на горизонтальной оси от $0$ до $\pi$ с 
    шагом $\pi / 6$, отметки подписать формулами как в Примере 3.  
    По вертикальной оси отметки $-3$, 0, $3$.

Вариант N31

Построить в одной координатной плоскости графики функций $f(x) = $
    $x^{2} + 4 x$, $g(x) = $
    $4 \sin{\left(2 x \right)}$ на 
    отрезке $\left[ 3 \pi / 2, \  5 \pi / 2\right]$, $f(x)$ фиолетовая 
    пунктирная линия, $g(x)$ желтая линия из точек. 
    Отметки на горизонтальной оси от $3 \pi / 2$ до $5 \pi / 2$ с 
    шагом $\pi / 4$, отметки подписать формулами как в Примере 3.  
    По вертикальной оси отметки $-4$, 0, $4$.

Вариант N32

Построить в одной координатной плоскости графики функций $f(x) = $
    $x$, $g(x) = $
    $3 \cos{\left(2 x \right)}$ на 
    отрезке $\left[ \pi / 2, \  \pi\right]$, $f(x)$ зеленая 
    пунктирная линия, $g(x)$ синяя пунктирная линия. 
    Отметки на горизонтальной оси от $\pi / 2$ до $\pi$ с 
    шагом $\pi / 4$, отметки подписать формулами как в Примере 3.  
    По вертикальной оси отметки $-3$, 0, $3$.

Вариант N33

Построить в одной координатной плоскости графики функций $f(x) = $
    $x^{2} + 4 x$, $g(x) = $
    $3 \cos{\left(3 x \right)}$ на 
    отрезке $\left[ 2 \pi / 3, \  5 \pi / 3\right]$, $f(x)$ красная 
    сплошная линия, $g(x)$ зеленая линия из точек. 
    Отметки на горизонтальной оси от $2 \pi / 3$ до $5 \pi / 3$ с 
    шагом $\pi / 6$, отметки подписать формулами как в Примере 3.  
    По вертикальной оси отметки $-3$, 0, $3$.

Вариант N34

Построить в одной координатной плоскости графики функций $f(x) = $
    $4 x$, $g(x) = $
    $4 \cos{\left(2 x \right)}$ на 
    отрезке $\left[ 0, \  3 \pi / 2\right]$, $f(x)$ фиолетовая 
    сплошная линия, $g(x)$ черная линия из точек. 
    Отметки на горизонтальной оси от $0$ до $3 \pi / 2$ с 
    шагом $\pi / 4$, отметки подписать формулами как в Примере 3.  
    По вертикальной оси отметки $-4$, 0, $4$.

Вариант N35

Построить в одной координатной плоскости графики функций $f(x) = $
    $2 x$, $g(x) = $
    $3 \cos{\left(2 x \right)}$ на 
    отрезке $\left[ 3 \pi / 2, \  3 \pi\right]$, $f(x)$ фиолетовая 
    сплошная линия, $g(x)$ черная линия из точек. 
    Отметки на горизонтальной оси от $3 \pi / 2$ до $3 \pi$ с 
    шагом $\pi / 4$, отметки подписать формулами как в Примере 3.  
    По вертикальной оси отметки $-3$, 0, $3$.

Вариант N36

Построить в одной координатной плоскости графики функций $f(x) = $
    $3 x + 4$, $g(x) = $
    $4 \cos{\left(4 x \right)}$ на 
    отрезке $\left[ 3 \pi / 4, \  5 \pi / 4\right]$, $f(x)$ желтая 
    сплошная линия, $g(x)$ синяя линия из точек. 
    Отметки на горизонтальной оси от $3 \pi / 4$ до $5 \pi / 4$ с 
    шагом $\pi / 8$, отметки подписать формулами как в Примере 3.  
    По вертикальной оси отметки $-4$, 0, $4$.

Вариант N37

Построить в одной координатной плоскости графики функций $f(x) = $
    $- 2 x$, $g(x) = $
    $3 \sin{\left(2 x \right)}$ на 
    отрезке $\left[ 3 \pi / 2, \  5 \pi / 2\right]$, $f(x)$ красная 
    линия из точек, $g(x)$ зеленая линия из точек. 
    Отметки на горизонтальной оси от $3 \pi / 2$ до $5 \pi / 2$ с 
    шагом $\pi / 4$, отметки подписать формулами как в Примере 3.  
    По вертикальной оси отметки $-3$, 0, $3$.

Вариант N38

Построить в одной координатной плоскости графики функций $f(x) = $
    $- 2 x$, $g(x) = $
    $2 \cos{\left(2 x \right)}$ на 
    отрезке $\left[ 0, \  3 \pi / 2\right]$, $f(x)$ зеленая 
    сплошная линия, $g(x)$ фиолетовая линия из точек. 
    Отметки на горизонтальной оси от $0$ до $3 \pi / 2$ с 
    шагом $\pi / 4$, отметки подписать формулами как в Примере 3.  
    По вертикальной оси отметки $-2$, 0, $2$.

Вариант N39

Построить в одной координатной плоскости графики функций $f(x) = $
    $- x^{2} + x$, $g(x) = $
    $3 \sin{\left(4 x \right)}$ на 
    отрезке $\left[ 3 \pi / 4, \  5 \pi / 4\right]$, $f(x)$ красная 
    пунктирная линия, $g(x)$ синяя пунктирная линия. 
    Отметки на горизонтальной оси от $3 \pi / 4$ до $5 \pi / 4$ с 
    шагом $\pi / 8$, отметки подписать формулами как в Примере 3.  
    По вертикальной оси отметки $-3$, 0, $3$.

Вариант N40

Построить в одной координатной плоскости графики функций $f(x) = $
    $x^{2} + 3 x - 2$, $g(x) = $
    $2 \cos{\left(2 x \right)}$ на 
    отрезке $\left[ \pi, \  2 \pi\right]$, $f(x)$ зеленая 
    линия из точек, $g(x)$ фиолетовая пунктирная линия. 
    Отметки на горизонтальной оси от $\pi$ до $2 \pi$ с 
    шагом $\pi / 4$, отметки подписать формулами как в Примере 3.  
    По вертикальной оси отметки $-2$, 0, $2$.

Вариант N41

Построить в одной координатной плоскости графики функций $f(x) = $
    $x^{2} - 2 x$, $g(x) = $
    $3 \cos{\left(4 x \right)}$ на 
    отрезке $\left[ \pi / 2, \  3 \pi / 4\right]$, $f(x)$ синяя 
    линия из точек, $g(x)$ черная пунктирная линия. 
    Отметки на горизонтальной оси от $\pi / 2$ до $3 \pi / 4$ с 
    шагом $\pi / 8$, отметки подписать формулами как в Примере 3.  
    По вертикальной оси отметки $-3$, 0, $3$.

Вариант N42

Построить в одной координатной плоскости графики функций $f(x) = $
    $- x - 4$, $g(x) = $
    $4 \sin{\left(4 x \right)}$ на 
    отрезке $\left[ \pi / 4, \  \pi / 2\right]$, $f(x)$ черная 
    пунктирная линия, $g(x)$ желтая линия из точек. 
    Отметки на горизонтальной оси от $\pi / 4$ до $\pi / 2$ с 
    шагом $\pi / 8$, отметки подписать формулами как в Примере 3.  
    По вертикальной оси отметки $-4$, 0, $4$.

Вариант N43

Построить в одной координатной плоскости графики функций $f(x) = $
    $x^{2} + 2 x$, $g(x) = $
    $4 \cos{\left(2 x \right)}$ на 
    отрезке $\left[ \pi, \  3 \pi / 2\right]$, $f(x)$ зеленая 
    сплошная линия, $g(x)$ фиолетовая пунктирная линия. 
    Отметки на горизонтальной оси от $\pi$ до $3 \pi / 2$ с 
    шагом $\pi / 4$, отметки подписать формулами как в Примере 3.  
    По вертикальной оси отметки $-4$, 0, $4$.

Вариант N44

Построить в одной координатной плоскости графики функций $f(x) = $
    $x^{2} - 4$, $g(x) = $
    $4 \sin{\left(4 x \right)}$ на 
    отрезке $\left[ 3 \pi / 4, \  \pi\right]$, $f(x)$ красная 
    линия из точек, $g(x)$ зеленая линия из точек. 
    Отметки на горизонтальной оси от $3 \pi / 4$ до $\pi$ с 
    шагом $\pi / 8$, отметки подписать формулами как в Примере 3.  
    По вертикальной оси отметки $-4$, 0, $4$.

Вариант N45

Построить в одной координатной плоскости графики функций $f(x) = $
    $- x^{2} - x - 4$, $g(x) = $
    $4 \sin{\left(2 x \right)}$ на 
    отрезке $\left[ \pi, \  5 \pi / 2\right]$, $f(x)$ зеленая 
    пунктирная линия, $g(x)$ фиолетовая линия из точек. 
    Отметки на горизонтальной оси от $\pi$ до $5 \pi / 2$ с 
    шагом $\pi / 4$, отметки подписать формулами как в Примере 3.  
    По вертикальной оси отметки $-4$, 0, $4$.

Вариант N46

Построить в одной координатной плоскости графики функций $f(x) = $
    $- x^{2} + 3 x$, $g(x) = $
    $2 \cos{\left(4 x \right)}$ на 
    отрезке $\left[ \pi / 4, \  \pi / 2\right]$, $f(x)$ фиолетовая 
    пунктирная линия, $g(x)$ желтая линия из точек. 
    Отметки на горизонтальной оси от $\pi / 4$ до $\pi / 2$ с 
    шагом $\pi / 8$, отметки подписать формулами как в Примере 3.  
    По вертикальной оси отметки $-2$, 0, $2$.

Вариант N47

Построить в одной координатной плоскости графики функций $f(x) = $
    $x^{2} - 4$, $g(x) = $
    $4 \sin{\left(3 x \right)}$ на 
    отрезке $\left[ 2 \pi / 3, \  5 \pi / 3\right]$, $f(x)$ синяя 
    пунктирная линия, $g(x)$ черная линия из точек. 
    Отметки на горизонтальной оси от $2 \pi / 3$ до $5 \pi / 3$ с 
    шагом $\pi / 6$, отметки подписать формулами как в Примере 3.  
    По вертикальной оси отметки $-4$, 0, $4$.

Вариант N48

Построить в одной координатной плоскости графики функций $f(x) = $
    $x^{2} - x + 2$, $g(x) = $
    $2 \sin{\left(2 x \right)}$ на 
    отрезке $\left[ \pi / 2, \  2 \pi\right]$, $f(x)$ зеленая 
    сплошная линия, $g(x)$ фиолетовая линия из точек. 
    Отметки на горизонтальной оси от $\pi / 2$ до $2 \pi$ с 
    шагом $\pi / 4$, отметки подписать формулами как в Примере 3.  
    По вертикальной оси отметки $-2$, 0, $2$.

Вариант N49

Построить в одной координатной плоскости графики функций $f(x) = $
    $- 2 x$, $g(x) = $
    $2 \sin{\left(4 x \right)}$ на 
    отрезке $\left[ 0, \  \pi / 2\right]$, $f(x)$ черная 
    пунктирная линия, $g(x)$ желтая линия из точек. 
    Отметки на горизонтальной оси от $0$ до $\pi / 2$ с 
    шагом $\pi / 8$, отметки подписать формулами как в Примере 3.  
    По вертикальной оси отметки $-2$, 0, $2$.

Вариант N50

Построить в одной координатной плоскости графики функций $f(x) = $
    $- x - 4$, $g(x) = $
    $4 \cos{\left(3 x \right)}$ на 
    отрезке $\left[ 2 \pi / 3, \  4 \pi / 3\right]$, $f(x)$ красная 
    пунктирная линия, $g(x)$ зеленая линия из точек. 
    Отметки на горизонтальной оси от $2 \pi / 3$ до $4 \pi / 3$ с 
    шагом $\pi / 6$, отметки подписать формулами как в Примере 3.  
    По вертикальной оси отметки $-4$, 0, $4$.

Вариант N51

Построить в одной координатной плоскости графики функций $f(x) = $
    $x^{2} - 2 x$, $g(x) = $
    $2 \cos{\left(3 x \right)}$ на 
    отрезке $\left[ 0, \  \pi\right]$, $f(x)$ синяя 
    пунктирная линия, $g(x)$ фиолетовая пунктирная линия. 
    Отметки на горизонтальной оси от $0$ до $\pi$ с 
    шагом $\pi / 6$, отметки подписать формулами как в Примере 3.  
    По вертикальной оси отметки $-2$, 0, $2$.

Вариант N52

Построить в одной координатной плоскости графики функций $f(x) = $
    $x^{2} - 4$, $g(x) = $
    $4 \sin{\left(2 x \right)}$ на 
    отрезке $\left[ 3 \pi / 2, \  2 \pi\right]$, $f(x)$ фиолетовая 
    сплошная линия, $g(x)$ черная пунктирная линия. 
    Отметки на горизонтальной оси от $3 \pi / 2$ до $2 \pi$ с 
    шагом $\pi / 4$, отметки подписать формулами как в Примере 3.  
    По вертикальной оси отметки $-4$, 0, $4$.

Вариант N53

Построить в одной координатной плоскости графики функций $f(x) = $
    $- x^{2} + x$, $g(x) = $
    $3 \cos{\left(4 x \right)}$ на 
    отрезке $\left[ \pi / 2, \  5 \pi / 4\right]$, $f(x)$ зеленая 
    линия из точек, $g(x)$ фиолетовая пунктирная линия. 
    Отметки на горизонтальной оси от $\pi / 2$ до $5 \pi / 4$ с 
    шагом $\pi / 8$, отметки подписать формулами как в Примере 3.  
    По вертикальной оси отметки $-3$, 0, $3$.

Вариант N54

Построить в одной координатной плоскости графики функций $f(x) = $
    $- x^{2} + 4 x$, $g(x) = $
    $4 \cos{\left(3 x \right)}$ на 
    отрезке $\left[ \pi / 3, \  4 \pi / 3\right]$, $f(x)$ зеленая 
    линия из точек, $g(x)$ синяя пунктирная линия. 
    Отметки на горизонтальной оси от $\pi / 3$ до $4 \pi / 3$ с 
    шагом $\pi / 6$, отметки подписать формулами как в Примере 3.  
    По вертикальной оси отметки $-4$, 0, $4$.

Вариант N55

Построить в одной координатной плоскости графики функций $f(x) = $
    $x^{2} + 4 x$, $g(x) = $
    $3 \cos{\left(4 x \right)}$ на 
    отрезке $\left[ 3 \pi / 4, \  \pi\right]$, $f(x)$ желтая 
    линия из точек, $g(x)$ зеленая линия из точек. 
    Отметки на горизонтальной оси от $3 \pi / 4$ до $\pi$ с 
    шагом $\pi / 8$, отметки подписать формулами как в Примере 3.  
    По вертикальной оси отметки $-3$, 0, $3$.

Вариант N56

Построить в одной координатной плоскости графики функций $f(x) = $
    $3 x$, $g(x) = $
    $3 \cos{\left(2 x \right)}$ на 
    отрезке $\left[ 3 \pi / 2, \  3 \pi\right]$, $f(x)$ черная 
    сплошная линия, $g(x)$ желтая линия из точек. 
    Отметки на горизонтальной оси от $3 \pi / 2$ до $3 \pi$ с 
    шагом $\pi / 4$, отметки подписать формулами как в Примере 3.  
    По вертикальной оси отметки $-3$, 0, $3$.

Вариант N57

Построить в одной координатной плоскости графики функций $f(x) = $
    $- x^{2} + 3 x$, $g(x) = $
    $2 \sin{\left(4 x \right)}$ на 
    отрезке $\left[ 3 \pi / 4, \  5 \pi / 4\right]$, $f(x)$ желтая 
    сплошная линия, $g(x)$ зеленая линия из точек. 
    Отметки на горизонтальной оси от $3 \pi / 4$ до $5 \pi / 4$ с 
    шагом $\pi / 8$, отметки подписать формулами как в Примере 3.  
    По вертикальной оси отметки $-2$, 0, $2$.

Вариант N58

Построить в одной координатной плоскости графики функций $f(x) = $
    $4 x$, $g(x) = $
    $4 \cos{\left(2 x \right)}$ на 
    отрезке $\left[ \pi, \  3 \pi / 2\right]$, $f(x)$ желтая 
    линия из точек, $g(x)$ зеленая линия из точек. 
    Отметки на горизонтальной оси от $\pi$ до $3 \pi / 2$ с 
    шагом $\pi / 4$, отметки подписать формулами как в Примере 3.  
    По вертикальной оси отметки $-4$, 0, $4$.

Вариант N59

Построить в одной координатной плоскости графики функций $f(x) = $
    $x^{2} - x + 4$, $g(x) = $
    $4 \cos{\left(3 x \right)}$ на 
    отрезке $\left[ \pi, \  5 \pi / 3\right]$, $f(x)$ синяя 
    пунктирная линия, $g(x)$ фиолетовая линия из точек. 
    Отметки на горизонтальной оси от $\pi$ до $5 \pi / 3$ с 
    шагом $\pi / 6$, отметки подписать формулами как в Примере 3.  
    По вертикальной оси отметки $-4$, 0, $4$.

Вариант N60

Построить в одной координатной плоскости графики функций $f(x) = $
    $x^{2} + 3 x + 3$, $g(x) = $
    $3 \cos{\left(4 x \right)}$ на 
    отрезке $\left[ \pi / 4, \  3 \pi / 4\right]$, $f(x)$ черная 
    пунктирная линия, $g(x)$ желтая пунктирная линия. 
    Отметки на горизонтальной оси от $\pi / 4$ до $3 \pi / 4$ с 
    шагом $\pi / 8$, отметки подписать формулами как в Примере 3.  
    По вертикальной оси отметки $-3$, 0, $3$.

Вариант N61

Построить в одной координатной плоскости графики функций $f(x) = $
    $- x$, $g(x) = $
    $2 \sin{\left(3 x \right)}$ на 
    отрезке $\left[ \pi / 3, \  \pi\right]$, $f(x)$ синяя 
    линия из точек, $g(x)$ черная пунктирная линия. 
    Отметки на горизонтальной оси от $\pi / 3$ до $\pi$ с 
    шагом $\pi / 6$, отметки подписать формулами как в Примере 3.  
    По вертикальной оси отметки $-2$, 0, $2$.

Вариант N62

Построить в одной координатной плоскости графики функций $f(x) = $
    $x^{2} - 2 x + 4$, $g(x) = $
    $4 \cos{\left(2 x \right)}$ на 
    отрезке $\left[ \pi / 2, \  3 \pi / 2\right]$, $f(x)$ красная 
    сплошная линия, $g(x)$ зеленая пунктирная линия. 
    Отметки на горизонтальной оси от $\pi / 2$ до $3 \pi / 2$ с 
    шагом $\pi / 4$, отметки подписать формулами как в Примере 3.  
    По вертикальной оси отметки $-4$, 0, $4$.

Вариант N63

Построить в одной координатной плоскости графики функций $f(x) = $
    $- x^{2} - x$, $g(x) = $
    $3 \cos{\left(2 x \right)}$ на 
    отрезке $\left[ \pi / 2, \  2 \pi\right]$, $f(x)$ фиолетовая 
    пунктирная линия, $g(x)$ желтая пунктирная линия. 
    Отметки на горизонтальной оси от $\pi / 2$ до $2 \pi$ с 
    шагом $\pi / 4$, отметки подписать формулами как в Примере 3.  
    По вертикальной оси отметки $-3$, 0, $3$.

Вариант N64

Построить в одной координатной плоскости графики функций $f(x) = $
    $x^{2} + x + 3$, $g(x) = $
    $3 \sin{\left(2 x \right)}$ на 
    отрезке $\left[ 3 \pi / 2, \  3 \pi\right]$, $f(x)$ красная 
    сплошная линия, $g(x)$ зеленая линия из точек. 
    Отметки на горизонтальной оси от $3 \pi / 2$ до $3 \pi$ с 
    шагом $\pi / 4$, отметки подписать формулами как в Примере 3.  
    По вертикальной оси отметки $-3$, 0, $3$.

Вариант N65

Построить в одной координатной плоскости графики функций $f(x) = $
    $- x$, $g(x) = $
    $4 \cos{\left(3 x \right)}$ на 
    отрезке $\left[ 0, \  \pi / 3\right]$, $f(x)$ желтая 
    сплошная линия, $g(x)$ зеленая пунктирная линия. 
    Отметки на горизонтальной оси от $0$ до $\pi / 3$ с 
    шагом $\pi / 6$, отметки подписать формулами как в Примере 3.  
    По вертикальной оси отметки $-4$, 0, $4$.

Вариант N66

Построить в одной координатной плоскости графики функций $f(x) = $
    $- x^{2} + 4 x$, $g(x) = $
    $3 \cos{\left(2 x \right)}$ на 
    отрезке $\left[ 3 \pi / 2, \  3 \pi\right]$, $f(x)$ черная 
    пунктирная линия, $g(x)$ желтая линия из точек. 
    Отметки на горизонтальной оси от $3 \pi / 2$ до $3 \pi$ с 
    шагом $\pi / 4$, отметки подписать формулами как в Примере 3.  
    По вертикальной оси отметки $-3$, 0, $3$.

Вариант N67

Построить в одной координатной плоскости графики функций $f(x) = $
    $- x^{2} + x + 2$, $g(x) = $
    $2 \sin{\left(2 x \right)}$ на 
    отрезке $\left[ \pi, \  3 \pi / 2\right]$, $f(x)$ фиолетовая 
    линия из точек, $g(x)$ желтая линия из точек. 
    Отметки на горизонтальной оси от $\pi$ до $3 \pi / 2$ с 
    шагом $\pi / 4$, отметки подписать формулами как в Примере 3.  
    По вертикальной оси отметки $-2$, 0, $2$.

Вариант N68

Построить в одной координатной плоскости графики функций $f(x) = $
    $x^{2} + 3 x$, $g(x) = $
    $4 \cos{\left(3 x \right)}$ на 
    отрезке $\left[ 0, \  \pi / 3\right]$, $f(x)$ синяя 
    пунктирная линия, $g(x)$ фиолетовая линия из точек. 
    Отметки на горизонтальной оси от $0$ до $\pi / 3$ с 
    шагом $\pi / 6$, отметки подписать формулами как в Примере 3.  
    По вертикальной оси отметки $-4$, 0, $4$.

Вариант N69

Построить в одной координатной плоскости графики функций $f(x) = $
    $x - 3$, $g(x) = $
    $3 \sin{\left(4 x \right)}$ на 
    отрезке $\left[ 0, \  3 \pi / 4\right]$, $f(x)$ красная 
    сплошная линия, $g(x)$ зеленая пунктирная линия. 
    Отметки на горизонтальной оси от $0$ до $3 \pi / 4$ с 
    шагом $\pi / 8$, отметки подписать формулами как в Примере 3.  
    По вертикальной оси отметки $-3$, 0, $3$.

Вариант N70

Построить в одной координатной плоскости графики функций $f(x) = $
    $2 - x^{2}$, $g(x) = $
    $2 \sin{\left(4 x \right)}$ на 
    отрезке $\left[ 0, \  \pi / 4\right]$, $f(x)$ красная 
    сплошная линия, $g(x)$ синяя пунктирная линия. 
    Отметки на горизонтальной оси от $0$ до $\pi / 4$ с 
    шагом $\pi / 8$, отметки подписать формулами как в Примере 3.  
    По вертикальной оси отметки $-2$, 0, $2$.

Вариант N71

Построить в одной координатной плоскости графики функций $f(x) = $
    $2 - 2 x$, $g(x) = $
    $2 \cos{\left(2 x \right)}$ на 
    отрезке $\left[ \pi / 2, \  3 \pi / 2\right]$, $f(x)$ зеленая 
    пунктирная линия, $g(x)$ синяя пунктирная линия. 
    Отметки на горизонтальной оси от $\pi / 2$ до $3 \pi / 2$ с 
    шагом $\pi / 4$, отметки подписать формулами как в Примере 3.  
    По вертикальной оси отметки $-2$, 0, $2$.

Вариант N72

Построить в одной координатной плоскости графики функций $f(x) = $
    $x^{2} + 4 x - 3$, $g(x) = $
    $3 \cos{\left(3 x \right)}$ на 
    отрезке $\left[ 2 \pi / 3, \  4 \pi / 3\right]$, $f(x)$ желтая 
    пунктирная линия, $g(x)$ зеленая пунктирная линия. 
    Отметки на горизонтальной оси от $2 \pi / 3$ до $4 \pi / 3$ с 
    шагом $\pi / 6$, отметки подписать формулами как в Примере 3.  
    По вертикальной оси отметки $-3$, 0, $3$.

Вариант N73

Построить в одной координатной плоскости графики функций $f(x) = $
    $x^{2} + 3 x - 4$, $g(x) = $
    $4 \sin{\left(2 x \right)}$ на 
    отрезке $\left[ 0, \  3 \pi / 2\right]$, $f(x)$ синяя 
    пунктирная линия, $g(x)$ черная линия из точек. 
    Отметки на горизонтальной оси от $0$ до $3 \pi / 2$ с 
    шагом $\pi / 4$, отметки подписать формулами как в Примере 3.  
    По вертикальной оси отметки $-4$, 0, $4$.

Вариант N74

Построить в одной координатной плоскости графики функций $f(x) = $
    $3 x - 3$, $g(x) = $
    $3 \cos{\left(4 x \right)}$ на 
    отрезке $\left[ 0, \  \pi / 2\right]$, $f(x)$ синяя 
    сплошная линия, $g(x)$ фиолетовая пунктирная линия. 
    Отметки на горизонтальной оси от $0$ до $\pi / 2$ с 
    шагом $\pi / 8$, отметки подписать формулами как в Примере 3.  
    По вертикальной оси отметки $-3$, 0, $3$.

Вариант N75

Построить в одной координатной плоскости графики функций $f(x) = $
    $x$, $g(x) = $
    $4 \cos{\left(4 x \right)}$ на 
    отрезке $\left[ \pi / 2, \  5 \pi / 4\right]$, $f(x)$ желтая 
    линия из точек, $g(x)$ зеленая линия из точек. 
    Отметки на горизонтальной оси от $\pi / 2$ до $5 \pi / 4$ с 
    шагом $\pi / 8$, отметки подписать формулами как в Примере 3.  
    По вертикальной оси отметки $-4$, 0, $4$.

Вариант N76

Построить в одной координатной плоскости графики функций $f(x) = $
    $x^{2} + 3$, $g(x) = $
    $3 \sin{\left(2 x \right)}$ на 
    отрезке $\left[ 0, \  \pi / 2\right]$, $f(x)$ зеленая 
    сплошная линия, $g(x)$ синяя линия из точек. 
    Отметки на горизонтальной оси от $0$ до $\pi / 2$ с 
    шагом $\pi / 4$, отметки подписать формулами как в Примере 3.  
    По вертикальной оси отметки $-3$, 0, $3$.

Вариант N77

Построить в одной координатной плоскости графики функций $f(x) = $
    $-4$, $g(x) = $
    $4 \sin{\left(4 x \right)}$ на 
    отрезке $\left[ \pi / 2, \  5 \pi / 4\right]$, $f(x)$ красная 
    линия из точек, $g(x)$ зеленая пунктирная линия. 
    Отметки на горизонтальной оси от $\pi / 2$ до $5 \pi / 4$ с 
    шагом $\pi / 8$, отметки подписать формулами как в Примере 3.  
    По вертикальной оси отметки $-4$, 0, $4$.

Вариант N78

Построить в одной координатной плоскости графики функций $f(x) = $
    $- x^{2} - x$, $g(x) = $
    $3 \cos{\left(2 x \right)}$ на 
    отрезке $\left[ 3 \pi / 2, \  2 \pi\right]$, $f(x)$ желтая 
    линия из точек, $g(x)$ синяя линия из точек. 
    Отметки на горизонтальной оси от $3 \pi / 2$ до $2 \pi$ с 
    шагом $\pi / 4$, отметки подписать формулами как в Примере 3.  
    По вертикальной оси отметки $-3$, 0, $3$.

Вариант N79

Построить в одной координатной плоскости графики функций $f(x) = $
    $2 - 2 x$, $g(x) = $
    $2 \cos{\left(2 x \right)}$ на 
    отрезке $\left[ 3 \pi / 2, \  5 \pi / 2\right]$, $f(x)$ черная 
    линия из точек, $g(x)$ зеленая линия из точек. 
    Отметки на горизонтальной оси от $3 \pi / 2$ до $5 \pi / 2$ с 
    шагом $\pi / 4$, отметки подписать формулами как в Примере 3.  
    По вертикальной оси отметки $-2$, 0, $2$.

Вариант N80

Построить в одной координатной плоскости графики функций $f(x) = $
    $x$, $g(x) = $
    $4 \sin{\left(4 x \right)}$ на 
    отрезке $\left[ 3 \pi / 4, \  5 \pi / 4\right]$, $f(x)$ синяя 
    пунктирная линия, $g(x)$ черная пунктирная линия. 
    Отметки на горизонтальной оси от $3 \pi / 4$ до $5 \pi / 4$ с 
    шагом $\pi / 8$, отметки подписать формулами как в Примере 3.  
    По вертикальной оси отметки $-4$, 0, $4$.

Вариант N81

Построить в одной координатной плоскости графики функций $f(x) = $
    $- x^{2} + 3 x + 2$, $g(x) = $
    $2 \sin{\left(4 x \right)}$ на 
    отрезке $\left[ 3 \pi / 4, \  3 \pi / 2\right]$, $f(x)$ черная 
    сплошная линия, $g(x)$ зеленая пунктирная линия. 
    Отметки на горизонтальной оси от $3 \pi / 4$ до $3 \pi / 2$ с 
    шагом $\pi / 8$, отметки подписать формулами как в Примере 3.  
    По вертикальной оси отметки $-2$, 0, $2$.

Вариант N82

Построить в одной координатной плоскости графики функций $f(x) = $
    $x^{2} + 4 x$, $g(x) = $
    $2 \cos{\left(4 x \right)}$ на 
    отрезке $\left[ \pi / 2, \  3 \pi / 4\right]$, $f(x)$ желтая 
    пунктирная линия, $g(x)$ зеленая линия из точек. 
    Отметки на горизонтальной оси от $\pi / 2$ до $3 \pi / 4$ с 
    шагом $\pi / 8$, отметки подписать формулами как в Примере 3.  
    По вертикальной оси отметки $-2$, 0, $2$.

Вариант N83

Построить в одной координатной плоскости графики функций $f(x) = $
    $- x$, $g(x) = $
    $3 \cos{\left(3 x \right)}$ на 
    отрезке $\left[ \pi, \  2 \pi\right]$, $f(x)$ фиолетовая 
    линия из точек, $g(x)$ черная пунктирная линия. 
    Отметки на горизонтальной оси от $\pi$ до $2 \pi$ с 
    шагом $\pi / 6$, отметки подписать формулами как в Примере 3.  
    По вертикальной оси отметки $-3$, 0, $3$.

Вариант N84

Построить в одной координатной плоскости графики функций $f(x) = $
    $- x^{2} - x + 3$, $g(x) = $
    $3 \cos{\left(3 x \right)}$ на 
    отрезке $\left[ \pi / 3, \  \pi\right]$, $f(x)$ черная 
    линия из точек, $g(x)$ зеленая линия из точек. 
    Отметки на горизонтальной оси от $\pi / 3$ до $\pi$ с 
    шагом $\pi / 6$, отметки подписать формулами как в Примере 3.  
    По вертикальной оси отметки $-3$, 0, $3$.

Вариант N85

Построить в одной координатной плоскости графики функций $f(x) = $
    $x^{2} - 2 x$, $g(x) = $
    $2 \cos{\left(2 x \right)}$ на 
    отрезке $\left[ \pi / 2, \  2 \pi\right]$, $f(x)$ желтая 
    линия из точек, $g(x)$ синяя пунктирная линия. 
    Отметки на горизонтальной оси от $\pi / 2$ до $2 \pi$ с 
    шагом $\pi / 4$, отметки подписать формулами как в Примере 3.  
    По вертикальной оси отметки $-2$, 0, $2$.

Вариант N86

Построить в одной координатной плоскости графики функций $f(x) = $
    $4 x + 4$, $g(x) = $
    $4 \sin{\left(3 x \right)}$ на 
    отрезке $\left[ \pi, \  2 \pi\right]$, $f(x)$ желтая 
    линия из точек, $g(x)$ синяя линия из точек. 
    Отметки на горизонтальной оси от $\pi$ до $2 \pi$ с 
    шагом $\pi / 6$, отметки подписать формулами как в Примере 3.  
    По вертикальной оси отметки $-4$, 0, $4$.

Вариант N87

Построить в одной координатной плоскости графики функций $f(x) = $
    $x^{2} + 3 x$, $g(x) = $
    $3 \sin{\left(4 x \right)}$ на 
    отрезке $\left[ 3 \pi / 4, \  5 \pi / 4\right]$, $f(x)$ желтая 
    пунктирная линия, $g(x)$ синяя пунктирная линия. 
    Отметки на горизонтальной оси от $3 \pi / 4$ до $5 \pi / 4$ с 
    шагом $\pi / 8$, отметки подписать формулами как в Примере 3.  
    По вертикальной оси отметки $-3$, 0, $3$.

Вариант N88

Построить в одной координатной плоскости графики функций $f(x) = $
    $- x^{2} + x - 4$, $g(x) = $
    $4 \cos{\left(3 x \right)}$ на 
    отрезке $\left[ 2 \pi / 3, \  4 \pi / 3\right]$, $f(x)$ желтая 
    сплошная линия, $g(x)$ зеленая пунктирная линия. 
    Отметки на горизонтальной оси от $2 \pi / 3$ до $4 \pi / 3$ с 
    шагом $\pi / 6$, отметки подписать формулами как в Примере 3.  
    По вертикальной оси отметки $-4$, 0, $4$.

Вариант N89

Построить в одной координатной плоскости графики функций $f(x) = $
    $- x^{2} + 3 x$, $g(x) = $
    $2 \cos{\left(4 x \right)}$ на 
    отрезке $\left[ \pi / 4, \  \pi\right]$, $f(x)$ желтая 
    сплошная линия, $g(x)$ синяя линия из точек. 
    Отметки на горизонтальной оси от $\pi / 4$ до $\pi$ с 
    шагом $\pi / 8$, отметки подписать формулами как в Примере 3.  
    По вертикальной оси отметки $-2$, 0, $2$.

Вариант N90

Построить в одной координатной плоскости графики функций $f(x) = $
    $3 x - 4$, $g(x) = $
    $4 \sin{\left(2 x \right)}$ на 
    отрезке $\left[ \pi, \  5 \pi / 2\right]$, $f(x)$ синяя 
    сплошная линия, $g(x)$ фиолетовая пунктирная линия. 
    Отметки на горизонтальной оси от $\pi$ до $5 \pi / 2$ с 
    шагом $\pi / 4$, отметки подписать формулами как в Примере 3.  
    По вертикальной оси отметки $-4$, 0, $4$.

Вариант N91

Построить в одной координатной плоскости графики функций $f(x) = $
    $- x^{2} - x + 2$, $g(x) = $
    $2 \sin{\left(2 x \right)}$ на 
    отрезке $\left[ \pi / 2, \  2 \pi\right]$, $f(x)$ фиолетовая 
    линия из точек, $g(x)$ черная линия из точек. 
    Отметки на горизонтальной оси от $\pi / 2$ до $2 \pi$ с 
    шагом $\pi / 4$, отметки подписать формулами как в Примере 3.  
    По вертикальной оси отметки $-2$, 0, $2$.

Вариант N92

Построить в одной координатной плоскости графики функций $f(x) = $
    $3 - 2 x$, $g(x) = $
    $3 \cos{\left(2 x \right)}$ на 
    отрезке $\left[ \pi / 2, \  3 \pi / 2\right]$, $f(x)$ синяя 
    линия из точек, $g(x)$ фиолетовая пунктирная линия. 
    Отметки на горизонтальной оси от $\pi / 2$ до $3 \pi / 2$ с 
    шагом $\pi / 4$, отметки подписать формулами как в Примере 3.  
    По вертикальной оси отметки $-3$, 0, $3$.

Вариант N93

Построить в одной координатной плоскости графики функций $f(x) = $
    $3 - x$, $g(x) = $
    $3 \cos{\left(4 x \right)}$ на 
    отрезке $\left[ 3 \pi / 4, \  5 \pi / 4\right]$, $f(x)$ красная 
    сплошная линия, $g(x)$ зеленая линия из точек. 
    Отметки на горизонтальной оси от $3 \pi / 4$ до $5 \pi / 4$ с 
    шагом $\pi / 8$, отметки подписать формулами как в Примере 3.  
    По вертикальной оси отметки $-3$, 0, $3$.

Вариант N94

Построить в одной координатной плоскости графики функций $f(x) = $
    $3 x$, $g(x) = $
    $4 \cos{\left(4 x \right)}$ на 
    отрезке $\left[ 3 \pi / 4, \  \pi\right]$, $f(x)$ фиолетовая 
    пунктирная линия, $g(x)$ желтая линия из точек. 
    Отметки на горизонтальной оси от $3 \pi / 4$ до $\pi$ с 
    шагом $\pi / 8$, отметки подписать формулами как в Примере 3.  
    По вертикальной оси отметки $-4$, 0, $4$.

Вариант N95

Построить в одной координатной плоскости графики функций $f(x) = $
    $- x^{2} - 2 x - 2$, $g(x) = $
    $2 \sin{\left(4 x \right)}$ на 
    отрезке $\left[ \pi / 2, \  5 \pi / 4\right]$, $f(x)$ синяя 
    линия из точек, $g(x)$ фиолетовая линия из точек. 
    Отметки на горизонтальной оси от $\pi / 2$ до $5 \pi / 4$ с 
    шагом $\pi / 8$, отметки подписать формулами как в Примере 3.  
    По вертикальной оси отметки $-2$, 0, $2$.

Вариант N96

Построить в одной координатной плоскости графики функций $f(x) = $
    $x^{2} - x$, $g(x) = $
    $4 \cos{\left(3 x \right)}$ на 
    отрезке $\left[ \pi / 3, \  4 \pi / 3\right]$, $f(x)$ синяя 
    сплошная линия, $g(x)$ фиолетовая линия из точек. 
    Отметки на горизонтальной оси от $\pi / 3$ до $4 \pi / 3$ с 
    шагом $\pi / 6$, отметки подписать формулами как в Примере 3.  
    По вертикальной оси отметки $-4$, 0, $4$.

Вариант N97

Построить в одной координатной плоскости графики функций $f(x) = $
    $- x - 2$, $g(x) = $
    $2 \cos{\left(2 x \right)}$ на 
    отрезке $\left[ 0, \  \pi\right]$, $f(x)$ красная 
    линия из точек, $g(x)$ зеленая пунктирная линия. 
    Отметки на горизонтальной оси от $0$ до $\pi$ с 
    шагом $\pi / 4$, отметки подписать формулами как в Примере 3.  
    По вертикальной оси отметки $-2$, 0, $2$.

Вариант N98

Построить в одной координатной плоскости графики функций $f(x) = $
    $x^{2} - x$, $g(x) = $
    $4 \cos{\left(3 x \right)}$ на 
    отрезке $\left[ 0, \  \pi / 3\right]$, $f(x)$ желтая 
    линия из точек, $g(x)$ синяя пунктирная линия. 
    Отметки на горизонтальной оси от $0$ до $\pi / 3$ с 
    шагом $\pi / 6$, отметки подписать формулами как в Примере 3.  
    По вертикальной оси отметки $-4$, 0, $4$.

Вариант N99

Построить в одной координатной плоскости графики функций $f(x) = $
    $- x^{2} - x$, $g(x) = $
    $2 \cos{\left(2 x \right)}$ на 
    отрезке $\left[ \pi, \  3 \pi / 2\right]$, $f(x)$ синяя 
    линия из точек, $g(x)$ черная линия из точек. 
    Отметки на горизонтальной оси от $\pi$ до $3 \pi / 2$ с 
    шагом $\pi / 4$, отметки подписать формулами как в Примере 3.  
    По вертикальной оси отметки $-2$, 0, $2$.

Вариант N100

Построить в одной координатной плоскости графики функций $f(x) = $
    $3 - x^{2}$, $g(x) = $
    $3 \cos{\left(3 x \right)}$ на 
    отрезке $\left[ \pi / 3, \  \pi\right]$, $f(x)$ синяя 
    сплошная линия, $g(x)$ черная пунктирная линия. 
    Отметки на горизонтальной оси от $\pi / 3$ до $\pi$ с 
    шагом $\pi / 6$, отметки подписать формулами как в Примере 3.  
    По вертикальной оси отметки $-3$, 0, $3$.

Вариант N101

Построить в одной координатной плоскости графики функций $f(x) = $
    $0$, $g(x) = $
    $4 \cos{\left(4 x \right)}$ на 
    отрезке $\left[ 0, \  \pi / 4\right]$, $f(x)$ зеленая 
    сплошная линия, $g(x)$ фиолетовая линия из точек. 
    Отметки на горизонтальной оси от $0$ до $\pi / 4$ с 
    шагом $\pi / 8$, отметки подписать формулами как в Примере 3.  
    По вертикальной оси отметки $-4$, 0, $4$.

Вариант N102

Построить в одной координатной плоскости графики функций $f(x) = $
    $x - 4$, $g(x) = $
    $4 \sin{\left(3 x \right)}$ на 
    отрезке $\left[ \pi, \  4 \pi / 3\right]$, $f(x)$ желтая 
    сплошная линия, $g(x)$ зеленая линия из точек. 
    Отметки на горизонтальной оси от $\pi$ до $4 \pi / 3$ с 
    шагом $\pi / 6$, отметки подписать формулами как в Примере 3.  
    По вертикальной оси отметки $-4$, 0, $4$.

Вариант N103

Построить в одной координатной плоскости графики функций $f(x) = $
    $-4$, $g(x) = $
    $4 \cos{\left(4 x \right)}$ на 
    отрезке $\left[ 0, \  \pi / 2\right]$, $f(x)$ синяя 
    пунктирная линия, $g(x)$ фиолетовая пунктирная линия. 
    Отметки на горизонтальной оси от $0$ до $\pi / 2$ с 
    шагом $\pi / 8$, отметки подписать формулами как в Примере 3.  
    По вертикальной оси отметки $-4$, 0, $4$.

Вариант N104

Построить в одной координатной плоскости графики функций $f(x) = $
    $x^{2} + x$, $g(x) = $
    $4 \cos{\left(4 x \right)}$ на 
    отрезке $\left[ 0, \  \pi / 4\right]$, $f(x)$ зеленая 
    сплошная линия, $g(x)$ синяя пунктирная линия. 
    Отметки на горизонтальной оси от $0$ до $\pi / 4$ с 
    шагом $\pi / 8$, отметки подписать формулами как в Примере 3.  
    По вертикальной оси отметки $-4$, 0, $4$.

Вариант N105

Построить в одной координатной плоскости графики функций $f(x) = $
    $x^{2} - x + 4$, $g(x) = $
    $4 \cos{\left(4 x \right)}$ на 
    отрезке $\left[ \pi / 4, \  3 \pi / 4\right]$, $f(x)$ красная 
    линия из точек, $g(x)$ синяя линия из точек. 
    Отметки на горизонтальной оси от $\pi / 4$ до $3 \pi / 4$ с 
    шагом $\pi / 8$, отметки подписать формулами как в Примере 3.  
    По вертикальной оси отметки $-4$, 0, $4$.

Вариант N106

Построить в одной координатной плоскости графики функций $f(x) = $
    $- x^{2} - x - 3$, $g(x) = $
    $3 \sin{\left(3 x \right)}$ на 
    отрезке $\left[ \pi / 3, \  2 \pi / 3\right]$, $f(x)$ зеленая 
    пунктирная линия, $g(x)$ фиолетовая пунктирная линия. 
    Отметки на горизонтальной оси от $\pi / 3$ до $2 \pi / 3$ с 
    шагом $\pi / 6$, отметки подписать формулами как в Примере 3.  
    По вертикальной оси отметки $-3$, 0, $3$.

Вариант N107

Построить в одной координатной плоскости графики функций $f(x) = $
    $- 2 x$, $g(x) = $
    $2 \cos{\left(3 x \right)}$ на 
    отрезке $\left[ 2 \pi / 3, \  5 \pi / 3\right]$, $f(x)$ красная 
    пунктирная линия, $g(x)$ синяя пунктирная линия. 
    Отметки на горизонтальной оси от $2 \pi / 3$ до $5 \pi / 3$ с 
    шагом $\pi / 6$, отметки подписать формулами как в Примере 3.  
    По вертикальной оси отметки $-2$, 0, $2$.

Вариант N108

Построить в одной координатной плоскости графики функций $f(x) = $
    $x^{2} - 2 x + 2$, $g(x) = $
    $2 \sin{\left(2 x \right)}$ на 
    отрезке $\left[ 0, \  \pi / 2\right]$, $f(x)$ черная 
    пунктирная линия, $g(x)$ зеленая пунктирная линия. 
    Отметки на горизонтальной оси от $0$ до $\pi / 2$ с 
    шагом $\pi / 4$, отметки подписать формулами как в Примере 3.  
    По вертикальной оси отметки $-2$, 0, $2$.

Вариант N109

Построить в одной координатной плоскости графики функций $f(x) = $
    $x^{2}$, $g(x) = $
    $3 \sin{\left(3 x \right)}$ на 
    отрезке $\left[ \pi, \  5 \pi / 3\right]$, $f(x)$ синяя 
    линия из точек, $g(x)$ фиолетовая пунктирная линия. 
    Отметки на горизонтальной оси от $\pi$ до $5 \pi / 3$ с 
    шагом $\pi / 6$, отметки подписать формулами как в Примере 3.  
    По вертикальной оси отметки $-3$, 0, $3$.

Вариант N110

Построить в одной координатной плоскости графики функций $f(x) = $
    $2 x + 3$, $g(x) = $
    $3 \sin{\left(4 x \right)}$ на 
    отрезке $\left[ 0, \  \pi / 4\right]$, $f(x)$ желтая 
    сплошная линия, $g(x)$ синяя линия из точек. 
    Отметки на горизонтальной оси от $0$ до $\pi / 4$ с 
    шагом $\pi / 8$, отметки подписать формулами как в Примере 3.  
    По вертикальной оси отметки $-3$, 0, $3$.

Вариант N111

Построить в одной координатной плоскости графики функций $f(x) = $
    $- x^{2} - 2 x$, $g(x) = $
    $2 \sin{\left(3 x \right)}$ на 
    отрезке $\left[ 2 \pi / 3, \  4 \pi / 3\right]$, $f(x)$ желтая 
    сплошная линия, $g(x)$ зеленая линия из точек. 
    Отметки на горизонтальной оси от $2 \pi / 3$ до $4 \pi / 3$ с 
    шагом $\pi / 6$, отметки подписать формулами как в Примере 3.  
    По вертикальной оси отметки $-2$, 0, $2$.

Вариант N112

Построить в одной координатной плоскости графики функций $f(x) = $
    $- x$, $g(x) = $
    $3 \cos{\left(3 x \right)}$ на 
    отрезке $\left[ \pi, \  2 \pi\right]$, $f(x)$ синяя 
    линия из точек, $g(x)$ фиолетовая пунктирная линия. 
    Отметки на горизонтальной оси от $\pi$ до $2 \pi$ с 
    шагом $\pi / 6$, отметки подписать формулами как в Примере 3.  
    По вертикальной оси отметки $-3$, 0, $3$.

Вариант N113

Построить в одной координатной плоскости графики функций $f(x) = $
    $- x^{2} + 2 x + 2$, $g(x) = $
    $2 \cos{\left(2 x \right)}$ на 
    отрезке $\left[ 3 \pi / 2, \  5 \pi / 2\right]$, $f(x)$ зеленая 
    пунктирная линия, $g(x)$ синяя линия из точек. 
    Отметки на горизонтальной оси от $3 \pi / 2$ до $5 \pi / 2$ с 
    шагом $\pi / 4$, отметки подписать формулами как в Примере 3.  
    По вертикальной оси отметки $-2$, 0, $2$.

Вариант N114

Построить в одной координатной плоскости графики функций $f(x) = $
    $x^{2} + 4 x$, $g(x) = $
    $2 \cos{\left(2 x \right)}$ на 
    отрезке $\left[ 3 \pi / 2, \  2 \pi\right]$, $f(x)$ красная 
    линия из точек, $g(x)$ синяя линия из точек. 
    Отметки на горизонтальной оси от $3 \pi / 2$ до $2 \pi$ с 
    шагом $\pi / 4$, отметки подписать формулами как в Примере 3.  
    По вертикальной оси отметки $-2$, 0, $2$.

Вариант N115

Построить в одной координатной плоскости графики функций $f(x) = $
    $2 x + 2$, $g(x) = $
    $2 \sin{\left(3 x \right)}$ на 
    отрезке $\left[ \pi / 3, \  4 \pi / 3\right]$, $f(x)$ красная 
    линия из точек, $g(x)$ зеленая пунктирная линия. 
    Отметки на горизонтальной оси от $\pi / 3$ до $4 \pi / 3$ с 
    шагом $\pi / 6$, отметки подписать формулами как в Примере 3.  
    По вертикальной оси отметки $-2$, 0, $2$.

Вариант N116

Построить в одной координатной плоскости графики функций $f(x) = $
    $- x^{2} - 4$, $g(x) = $
    $4 \sin{\left(2 x \right)}$ на 
    отрезке $\left[ 3 \pi / 2, \  2 \pi\right]$, $f(x)$ фиолетовая 
    сплошная линия, $g(x)$ желтая пунктирная линия. 
    Отметки на горизонтальной оси от $3 \pi / 2$ до $2 \pi$ с 
    шагом $\pi / 4$, отметки подписать формулами как в Примере 3.  
    По вертикальной оси отметки $-4$, 0, $4$.

Вариант N117

Построить в одной координатной плоскости графики функций $f(x) = $
    $- x^{2} + 4 x + 3$, $g(x) = $
    $3 \sin{\left(4 x \right)}$ на 
    отрезке $\left[ 3 \pi / 4, \  3 \pi / 2\right]$, $f(x)$ красная 
    сплошная линия, $g(x)$ зеленая пунктирная линия. 
    Отметки на горизонтальной оси от $3 \pi / 4$ до $3 \pi / 2$ с 
    шагом $\pi / 8$, отметки подписать формулами как в Примере 3.  
    По вертикальной оси отметки $-3$, 0, $3$.

Вариант N118

Построить в одной координатной плоскости графики функций $f(x) = $
    $- x^{2} + x$, $g(x) = $
    $4 \sin{\left(2 x \right)}$ на 
    отрезке $\left[ 3 \pi / 2, \  5 \pi / 2\right]$, $f(x)$ зеленая 
    линия из точек, $g(x)$ синяя пунктирная линия. 
    Отметки на горизонтальной оси от $3 \pi / 2$ до $5 \pi / 2$ с 
    шагом $\pi / 4$, отметки подписать формулами как в Примере 3.  
    По вертикальной оси отметки $-4$, 0, $4$.

Вариант N119

Построить в одной координатной плоскости графики функций $f(x) = $
    $0$, $g(x) = $
    $2 \cos{\left(2 x \right)}$ на 
    отрезке $\left[ \pi, \  3 \pi / 2\right]$, $f(x)$ красная 
    пунктирная линия, $g(x)$ зеленая линия из точек. 
    Отметки на горизонтальной оси от $\pi$ до $3 \pi / 2$ с 
    шагом $\pi / 4$, отметки подписать формулами как в Примере 3.  
    По вертикальной оси отметки $-2$, 0, $2$.

Вариант N120

Построить в одной координатной плоскости графики функций $f(x) = $
    $- x^{2} + 2 x$, $g(x) = $
    $4 \sin{\left(3 x \right)}$ на 
    отрезке $\left[ 2 \pi / 3, \  4 \pi / 3\right]$, $f(x)$ зеленая 
    сплошная линия, $g(x)$ синяя пунктирная линия. 
    Отметки на горизонтальной оси от $2 \pi / 3$ до $4 \pi / 3$ с 
    шагом $\pi / 6$, отметки подписать формулами как в Примере 3.  
    По вертикальной оси отметки $-4$, 0, $4$.

Вариант N121

Построить в одной координатной плоскости графики функций $f(x) = $
    $- x^{2} + 3 x - 4$, $g(x) = $
    $4 \sin{\left(2 x \right)}$ на 
    отрезке $\left[ 3 \pi / 2, \  2 \pi\right]$, $f(x)$ фиолетовая 
    сплошная линия, $g(x)$ черная пунктирная линия. 
    Отметки на горизонтальной оси от $3 \pi / 2$ до $2 \pi$ с 
    шагом $\pi / 4$, отметки подписать формулами как в Примере 3.  
    По вертикальной оси отметки $-4$, 0, $4$.

Вариант N122

Построить в одной координатной плоскости графики функций $f(x) = $
    $- 2 x$, $g(x) = $
    $3 \cos{\left(2 x \right)}$ на 
    отрезке $\left[ \pi, \  3 \pi / 2\right]$, $f(x)$ черная 
    линия из точек, $g(x)$ желтая пунктирная линия. 
    Отметки на горизонтальной оси от $\pi$ до $3 \pi / 2$ с 
    шагом $\pi / 4$, отметки подписать формулами как в Примере 3.  
    По вертикальной оси отметки $-3$, 0, $3$.

Вариант N123

Построить в одной координатной плоскости графики функций $f(x) = $
    $- x^{2} - x - 4$, $g(x) = $
    $4 \sin{\left(4 x \right)}$ на 
    отрезке $\left[ \pi / 2, \  5 \pi / 4\right]$, $f(x)$ черная 
    пунктирная линия, $g(x)$ зеленая пунктирная линия. 
    Отметки на горизонтальной оси от $\pi / 2$ до $5 \pi / 4$ с 
    шагом $\pi / 8$, отметки подписать формулами как в Примере 3.  
    По вертикальной оси отметки $-4$, 0, $4$.

Вариант N124

Построить в одной координатной плоскости графики функций $f(x) = $
    $x^{2} + 4 x - 4$, $g(x) = $
    $4 \cos{\left(4 x \right)}$ на 
    отрезке $\left[ 0, \  \pi / 2\right]$, $f(x)$ синяя 
    линия из точек, $g(x)$ черная пунктирная линия. 
    Отметки на горизонтальной оси от $0$ до $\pi / 2$ с 
    шагом $\pi / 8$, отметки подписать формулами как в Примере 3.  
    По вертикальной оси отметки $-4$, 0, $4$.

Вариант N125

Построить в одной координатной плоскости графики функций $f(x) = $
    $4 x$, $g(x) = $
    $3 \cos{\left(2 x \right)}$ на 
    отрезке $\left[ 0, \  3 \pi / 2\right]$, $f(x)$ желтая 
    линия из точек, $g(x)$ синяя пунктирная линия. 
    Отметки на горизонтальной оси от $0$ до $3 \pi / 2$ с 
    шагом $\pi / 4$, отметки подписать формулами как в Примере 3.  
    По вертикальной оси отметки $-3$, 0, $3$.

Вариант N126

Построить в одной координатной плоскости графики функций $f(x) = $
    $x$, $g(x) = $
    $2 \cos{\left(4 x \right)}$ на 
    отрезке $\left[ 3 \pi / 4, \  \pi\right]$, $f(x)$ желтая 
    пунктирная линия, $g(x)$ синяя линия из точек. 
    Отметки на горизонтальной оси от $3 \pi / 4$ до $\pi$ с 
    шагом $\pi / 8$, отметки подписать формулами как в Примере 3.  
    По вертикальной оси отметки $-2$, 0, $2$.

Вариант N127

Построить в одной координатной плоскости графики функций $f(x) = $
    $4 - x$, $g(x) = $
    $4 \cos{\left(4 x \right)}$ на 
    отрезке $\left[ \pi / 4, \  3 \pi / 4\right]$, $f(x)$ желтая 
    сплошная линия, $g(x)$ синяя пунктирная линия. 
    Отметки на горизонтальной оси от $\pi / 4$ до $3 \pi / 4$ с 
    шагом $\pi / 8$, отметки подписать формулами как в Примере 3.  
    По вертикальной оси отметки $-4$, 0, $4$.

Вариант N128

Построить в одной координатной плоскости графики функций $f(x) = $
    $x^{2} + 3 x$, $g(x) = $
    $3 \cos{\left(3 x \right)}$ на 
    отрезке $\left[ 2 \pi / 3, \  \pi\right]$, $f(x)$ черная 
    сплошная линия, $g(x)$ зеленая пунктирная линия. 
    Отметки на горизонтальной оси от $2 \pi / 3$ до $\pi$ с 
    шагом $\pi / 6$, отметки подписать формулами как в Примере 3.  
    По вертикальной оси отметки $-3$, 0, $3$.

Вариант N129

Построить в одной координатной плоскости графики функций $f(x) = $
    $x^{2} + 2 x - 3$, $g(x) = $
    $3 \cos{\left(4 x \right)}$ на 
    отрезке $\left[ \pi / 2, \  \pi\right]$, $f(x)$ синяя 
    линия из точек, $g(x)$ черная пунктирная линия. 
    Отметки на горизонтальной оси от $\pi / 2$ до $\pi$ с 
    шагом $\pi / 8$, отметки подписать формулами как в Примере 3.  
    По вертикальной оси отметки $-3$, 0, $3$.

Вариант N130

Построить в одной координатной плоскости графики функций $f(x) = $
    $- x^{2} + 4 x - 2$, $g(x) = $
    $2 \sin{\left(2 x \right)}$ на 
    отрезке $\left[ \pi, \  5 \pi / 2\right]$, $f(x)$ желтая 
    сплошная линия, $g(x)$ зеленая пунктирная линия. 
    Отметки на горизонтальной оси от $\pi$ до $5 \pi / 2$ с 
    шагом $\pi / 4$, отметки подписать формулами как в Примере 3.  
    По вертикальной оси отметки $-2$, 0, $2$.

Вариант N131

Построить в одной координатной плоскости графики функций $f(x) = $
    $- x^{2} - 2 x$, $g(x) = $
    $2 \sin{\left(4 x \right)}$ на 
    отрезке $\left[ \pi / 4, \  3 \pi / 4\right]$, $f(x)$ черная 
    пунктирная линия, $g(x)$ зеленая линия из точек. 
    Отметки на горизонтальной оси от $\pi / 4$ до $3 \pi / 4$ с 
    шагом $\pi / 8$, отметки подписать формулами как в Примере 3.  
    По вертикальной оси отметки $-2$, 0, $2$.

Вариант N132

Построить в одной координатной плоскости графики функций $f(x) = $
    $- x^{2} - x - 2$, $g(x) = $
    $2 \cos{\left(4 x \right)}$ на 
    отрезке $\left[ 0, \  \pi / 2\right]$, $f(x)$ зеленая 
    линия из точек, $g(x)$ синяя линия из точек. 
    Отметки на горизонтальной оси от $0$ до $\pi / 2$ с 
    шагом $\pi / 8$, отметки подписать формулами как в Примере 3.  
    По вертикальной оси отметки $-2$, 0, $2$.

Вариант N133

Построить в одной координатной плоскости графики функций $f(x) = $
    $3 x + 2$, $g(x) = $
    $2 \sin{\left(2 x \right)}$ на 
    отрезке $\left[ \pi / 2, \  2 \pi\right]$, $f(x)$ синяя 
    сплошная линия, $g(x)$ фиолетовая пунктирная линия. 
    Отметки на горизонтальной оси от $\pi / 2$ до $2 \pi$ с 
    шагом $\pi / 4$, отметки подписать формулами как в Примере 3.  
    По вертикальной оси отметки $-2$, 0, $2$.

Вариант N134

Построить в одной координатной плоскости графики функций $f(x) = $
    $x + 2$, $g(x) = $
    $2 \sin{\left(3 x \right)}$ на 
    отрезке $\left[ 0, \  \pi / 3\right]$, $f(x)$ черная 
    линия из точек, $g(x)$ желтая линия из точек. 
    Отметки на горизонтальной оси от $0$ до $\pi / 3$ с 
    шагом $\pi / 6$, отметки подписать формулами как в Примере 3.  
    По вертикальной оси отметки $-2$, 0, $2$.

Вариант N135

Построить в одной координатной плоскости графики функций $f(x) = $
    $4 x + 4$, $g(x) = $
    $4 \cos{\left(2 x \right)}$ на 
    отрезке $\left[ 3 \pi / 2, \  5 \pi / 2\right]$, $f(x)$ черная 
    пунктирная линия, $g(x)$ желтая линия из точек. 
    Отметки на горизонтальной оси от $3 \pi / 2$ до $5 \pi / 2$ с 
    шагом $\pi / 4$, отметки подписать формулами как в Примере 3.  
    По вертикальной оси отметки $-4$, 0, $4$.

Вариант N136

Построить в одной координатной плоскости графики функций $f(x) = $
    $x$, $g(x) = $
    $2 \cos{\left(2 x \right)}$ на 
    отрезке $\left[ \pi, \  3 \pi / 2\right]$, $f(x)$ желтая 
    сплошная линия, $g(x)$ синяя пунктирная линия. 
    Отметки на горизонтальной оси от $\pi$ до $3 \pi / 2$ с 
    шагом $\pi / 4$, отметки подписать формулами как в Примере 3.  
    По вертикальной оси отметки $-2$, 0, $2$.

Вариант N137

Построить в одной координатной плоскости графики функций $f(x) = $
    $- x^{2} + 4 x$, $g(x) = $
    $3 \cos{\left(3 x \right)}$ на 
    отрезке $\left[ \pi, \  4 \pi / 3\right]$, $f(x)$ черная 
    линия из точек, $g(x)$ желтая линия из точек. 
    Отметки на горизонтальной оси от $\pi$ до $4 \pi / 3$ с 
    шагом $\pi / 6$, отметки подписать формулами как в Примере 3.  
    По вертикальной оси отметки $-3$, 0, $3$.

Вариант N138

Построить в одной координатной плоскости графики функций $f(x) = $
    $- x^{2} - x + 4$, $g(x) = $
    $4 \sin{\left(4 x \right)}$ на 
    отрезке $\left[ \pi / 4, \  \pi\right]$, $f(x)$ желтая 
    линия из точек, $g(x)$ синяя линия из точек. 
    Отметки на горизонтальной оси от $\pi / 4$ до $\pi$ с 
    шагом $\pi / 8$, отметки подписать формулами как в Примере 3.  
    По вертикальной оси отметки $-4$, 0, $4$.

Вариант N139

Построить в одной координатной плоскости графики функций $f(x) = $
    $x^{2} - 2 x + 3$, $g(x) = $
    $3 \sin{\left(4 x \right)}$ на 
    отрезке $\left[ \pi / 4, \  \pi\right]$, $f(x)$ фиолетовая 
    пунктирная линия, $g(x)$ черная пунктирная линия. 
    Отметки на горизонтальной оси от $\pi / 4$ до $\pi$ с 
    шагом $\pi / 8$, отметки подписать формулами как в Примере 3.  
    По вертикальной оси отметки $-3$, 0, $3$.

Вариант N140

Построить в одной координатной плоскости графики функций $f(x) = $
    $- x^{2} - x$, $g(x) = $
    $3 \cos{\left(3 x \right)}$ на 
    отрезке $\left[ 2 \pi / 3, \  \pi\right]$, $f(x)$ черная 
    линия из точек, $g(x)$ зеленая линия из точек. 
    Отметки на горизонтальной оси от $2 \pi / 3$ до $\pi$ с 
    шагом $\pi / 6$, отметки подписать формулами как в Примере 3.  
    По вертикальной оси отметки $-3$, 0, $3$.

Вариант N141

Построить в одной координатной плоскости графики функций $f(x) = $
    $- x^{2} - x$, $g(x) = $
    $4 \sin{\left(4 x \right)}$ на 
    отрезке $\left[ 0, \  \pi / 2\right]$, $f(x)$ синяя 
    сплошная линия, $g(x)$ черная линия из точек. 
    Отметки на горизонтальной оси от $0$ до $\pi / 2$ с 
    шагом $\pi / 8$, отметки подписать формулами как в Примере 3.  
    По вертикальной оси отметки $-4$, 0, $4$.

Вариант N142

Построить в одной координатной плоскости графики функций $f(x) = $
    $0$, $g(x) = $
    $3 \cos{\left(4 x \right)}$ на 
    отрезке $\left[ \pi / 2, \  3 \pi / 4\right]$, $f(x)$ красная 
    сплошная линия, $g(x)$ зеленая линия из точек. 
    Отметки на горизонтальной оси от $\pi / 2$ до $3 \pi / 4$ с 
    шагом $\pi / 8$, отметки подписать формулами как в Примере 3.  
    По вертикальной оси отметки $-3$, 0, $3$.

Вариант N143

Построить в одной координатной плоскости графики функций $f(x) = $
    $2 x + 3$, $g(x) = $
    $3 \sin{\left(4 x \right)}$ на 
    отрезке $\left[ 0, \  \pi / 4\right]$, $f(x)$ зеленая 
    линия из точек, $g(x)$ фиолетовая пунктирная линия. 
    Отметки на горизонтальной оси от $0$ до $\pi / 4$ с 
    шагом $\pi / 8$, отметки подписать формулами как в Примере 3.  
    По вертикальной оси отметки $-3$, 0, $3$.

Вариант N144

Построить в одной координатной плоскости графики функций $f(x) = $
    $x^{2}$, $g(x) = $
    $2 \sin{\left(2 x \right)}$ на 
    отрезке $\left[ \pi, \  2 \pi\right]$, $f(x)$ фиолетовая 
    линия из точек, $g(x)$ желтая линия из точек. 
    Отметки на горизонтальной оси от $\pi$ до $2 \pi$ с 
    шагом $\pi / 4$, отметки подписать формулами как в Примере 3.  
    По вертикальной оси отметки $-2$, 0, $2$.

Вариант N145

Построить в одной координатной плоскости графики функций $f(x) = $
    $4 - 2 x$, $g(x) = $
    $4 \sin{\left(2 x \right)}$ на 
    отрезке $\left[ \pi, \  3 \pi / 2\right]$, $f(x)$ фиолетовая 
    линия из точек, $g(x)$ желтая линия из точек. 
    Отметки на горизонтальной оси от $\pi$ до $3 \pi / 2$ с 
    шагом $\pi / 4$, отметки подписать формулами как в Примере 3.  
    По вертикальной оси отметки $-4$, 0, $4$.

Вариант N146

Построить в одной координатной плоскости графики функций $f(x) = $
    $- x^{2} + 3 x$, $g(x) = $
    $4 \cos{\left(4 x \right)}$ на 
    отрезке $\left[ 0, \  3 \pi / 4\right]$, $f(x)$ черная 
    пунктирная линия, $g(x)$ желтая линия из точек. 
    Отметки на горизонтальной оси от $0$ до $3 \pi / 4$ с 
    шагом $\pi / 8$, отметки подписать формулами как в Примере 3.  
    По вертикальной оси отметки $-4$, 0, $4$.

Вариант N147

Построить в одной координатной плоскости графики функций $f(x) = $
    $3 x + 2$, $g(x) = $
    $2 \sin{\left(2 x \right)}$ на 
    отрезке $\left[ \pi / 2, \  2 \pi\right]$, $f(x)$ красная 
    сплошная линия, $g(x)$ зеленая пунктирная линия. 
    Отметки на горизонтальной оси от $\pi / 2$ до $2 \pi$ с 
    шагом $\pi / 4$, отметки подписать формулами как в Примере 3.  
    По вертикальной оси отметки $-2$, 0, $2$.

Вариант N148

Построить в одной координатной плоскости графики функций $f(x) = $
    $3 x - 4$, $g(x) = $
    $4 \sin{\left(3 x \right)}$ на 
    отрезке $\left[ 0, \  \pi\right]$, $f(x)$ красная 
    линия из точек, $g(x)$ зеленая линия из точек. 
    Отметки на горизонтальной оси от $0$ до $\pi$ с 
    шагом $\pi / 6$, отметки подписать формулами как в Примере 3.  
    По вертикальной оси отметки $-4$, 0, $4$.

Вариант N149

Построить в одной координатной плоскости графики функций $f(x) = $
    $4 - 2 x$, $g(x) = $
    $4 \cos{\left(4 x \right)}$ на 
    отрезке $\left[ 3 \pi / 4, \  5 \pi / 4\right]$, $f(x)$ фиолетовая 
    линия из точек, $g(x)$ желтая пунктирная линия. 
    Отметки на горизонтальной оси от $3 \pi / 4$ до $5 \pi / 4$ с 
    шагом $\pi / 8$, отметки подписать формулами как в Примере 3.  
    По вертикальной оси отметки $-4$, 0, $4$.

Вариант N150

Построить в одной координатной плоскости графики функций $f(x) = $
    $- x^{2} - x + 4$, $g(x) = $
    $4 \sin{\left(3 x \right)}$ на 
    отрезке $\left[ \pi / 3, \  4 \pi / 3\right]$, $f(x)$ фиолетовая 
    пунктирная линия, $g(x)$ желтая пунктирная линия. 
    Отметки на горизонтальной оси от $\pi / 3$ до $4 \pi / 3$ с 
    шагом $\pi / 6$, отметки подписать формулами как в Примере 3.  
    По вертикальной оси отметки $-4$, 0, $4$.

\end{document}