 \documentclass[11pt]{report}

\usepackage[T2A]{fontenc}

\usepackage[utf8]{inputenc}

\usepackage[russian]{babel}

\usepackage{amsmath,amssymb}

\usepackage{graphicx}

\graphicspath{ {d:/HSE/OR/CW/CW5pict/} }

\begin{document}

\pagestyle{empty}

{\bf Индивидуальное задание.}

 Вариант N1

Даны точки в пространстве
$F(-1, 2, 15)$, $K(16, 6, 12)$, $P(8, -2, 7)$ и
$W(12, 1, 10)$.

Найти угол между прямой $PW$ и плоскостью $z = 0$, угол между $PK$ и $KF$, угол между плоскостями $PWK$ 
и $WKF$.

Составить уравнение: 

a) плоскости, параллельной $PWK$ и проходящей через $F$,

b) плоскости, перпендикулярной $PWK$ и проходящей через $P$ и $K$,

c) прямой, перпендикулярной $PWK$ и проходящей через $K$.

Вариант N2

Даны точки в пространстве
$C(5, 7, -1)$, $M(4, -3, -13)$, $R(0, 6, -5)$ и
$W(-3, 4, -6)$.

Найти угол между прямой $WR$ и плоскостью $z = 0$, угол между $WC$ и $CM$, угол между плоскостями $WRC$ 
и $RCM$.

Составить уравнение: 

a) плоскости, параллельной $WRC$ и проходящей через $M$,

b) плоскости, перпендикулярной $WRC$ и проходящей через $W$ и $C$,

c) прямой, перпендикулярной $WRC$ и проходящей через $C$.

Вариант N3

Даны точки в пространстве
$B(-13, -4, 9)$, $H(-3, -8, -1)$, $P(1, -5, 0)$ и
$U(-7, -8, -3)$.

Найти угол между прямой $UH$ и плоскостью $z = 0$, угол между $UP$ и $PB$, угол между плоскостями $UHP$ 
и $HPB$.

Составить уравнение: 

a) плоскости, параллельной $UHP$ и проходящей через $B$,

b) плоскости, перпендикулярной $UHP$ и проходящей через $U$ и $P$,

c) прямой, перпендикулярной $UHP$ и проходящей через $P$.

Вариант N4

Даны точки в пространстве
$A(1, 0, -5)$, $M(4, 4, -3)$, $R(1, -1, -6)$ и
$W(-1, 2, -9)$.

Найти угол между прямой $RA$ и плоскостью $z = 0$, угол между $RM$ и $MW$, угол между плоскостями $RAM$ 
и $AMW$.

Составить уравнение: 

a) плоскости, параллельной $RAM$ и проходящей через $W$,

b) плоскости, перпендикулярной $RAM$ и проходящей через $R$ и $M$,

c) прямой, перпендикулярной $RAM$ и проходящей через $M$.

Вариант N5

Даны точки в пространстве
$K(-21, 5, 11)$, $P(-9, 9, -3)$, $V(-5, 11, 1)$ и
$W(-2, 16, 5)$.

Найти угол между прямой $PV$ и плоскостью $z = 0$, угол между $PW$ и $WK$, угол между плоскостями $PVW$ 
и $VWK$.

Составить уравнение: 

a) плоскости, параллельной $PVW$ и проходящей через $K$,

b) плоскости, перпендикулярной $PVW$ и проходящей через $P$ и $W$,

c) прямой, перпендикулярной $PVW$ и проходящей через $W$.

Вариант N6

Даны точки в пространстве
$C(7, -5, 4)$, $H(9, 5, 11)$, $M(3, -1, 4)$ и
$U(7, 3, 8)$.

Найти угол между прямой $MU$ и плоскостью $z = 0$, угол между $MH$ и $HC$, угол между плоскостями $MUH$ 
и $UHC$.

Составить уравнение: 

a) плоскости, параллельной $MUH$ и проходящей через $C$,

b) плоскости, перпендикулярной $MUH$ и проходящей через $M$ и $H$,

c) прямой, перпендикулярной $MUH$ и проходящей через $H$.

Вариант N7

Даны точки в пространстве
$C(-14, 1, -4)$, $K(1, 4, 0)$, $R(-1, 0, -5)$ и
$W(-2, 0, -8)$.

Найти угол между прямой $WR$ и плоскостью $z = 0$, угол между $WK$ и $KC$, угол между плоскостями $WRK$ 
и $RKC$.

Составить уравнение: 

a) плоскости, параллельной $WRK$ и проходящей через $C$,

b) плоскости, перпендикулярной $WRK$ и проходящей через $W$ и $K$,

c) прямой, перпендикулярной $WRK$ и проходящей через $K$.

Вариант N8

Даны точки в пространстве
$A(-2, -2, -4)$, $B(2, 2, 0)$, $H(-5, -11, 3)$ и
$U(-5, -3, -5)$.

Найти угол между прямой $UA$ и плоскостью $z = 0$, угол между $UB$ и $BH$, угол между плоскостями $UAB$ 
и $ABH$.

Составить уравнение: 

a) плоскости, параллельной $UAB$ и проходящей через $H$,

b) плоскости, перпендикулярной $UAB$ и проходящей через $U$ и $B$,

c) прямой, перпендикулярной $UAB$ и проходящей через $B$.

Вариант N9

Даны точки в пространстве
$C(-1, -1, -1)$, $D(5, 3, 0)$, $M(3, 0, -1)$ и
$Q(0, -5, 9)$.

Найти угол между прямой $CM$ и плоскостью $z = 0$, угол между $CD$ и $DQ$, угол между плоскостями $CMD$ 
и $MDQ$.

Составить уравнение: 

a) плоскости, параллельной $CMD$ и проходящей через $Q$,

b) плоскости, перпендикулярной $CMD$ и проходящей через $C$ и $D$,

c) прямой, перпендикулярной $CMD$ и проходящей через $D$.

Вариант N10

Даны точки в пространстве
$F(0, 9, 6)$, $H(-3, 13, 2)$, $V(-2, 8, 3)$ и
$W(3, 10, 8)$.

Найти угол между прямой $VF$ и плоскостью $z = 0$, угол между $VW$ и $WH$, угол между плоскостями $VFW$ 
и $FWH$.

Составить уравнение: 

a) плоскости, параллельной $VFW$ и проходящей через $H$,

b) плоскости, перпендикулярной $VFW$ и проходящей через $V$ и $W$,

c) прямой, перпендикулярной $VFW$ и проходящей через $W$.

Вариант N11

Даны точки в пространстве
$A(-6, -2, -3)$, $H(-7, -8, -5)$, $K(-8, -7, -6)$ и
$P(-7, -5, -5)$.

Найти угол между прямой $KP$ и плоскостью $z = 0$, угол между $KA$ и $AH$, угол между плоскостями $KPA$ 
и $PAH$.

Составить уравнение: 

a) плоскости, параллельной $KPA$ и проходящей через $H$,

b) плоскости, перпендикулярной $KPA$ и проходящей через $K$ и $A$,

c) прямой, перпендикулярной $KPA$ и проходящей через $A$.

Вариант N12

Даны точки в пространстве
$B(-3, 12, -9)$, $P(5, -7, -4)$, $Q(-7, 9, -9)$ и
$U(-2, 14, -5)$.

Найти угол между прямой $QB$ и плоскостью $z = 0$, угол между $QU$ и $UP$, угол между плоскостями $QBU$ 
и $BUP$.

Составить уравнение: 

a) плоскости, параллельной $QBU$ и проходящей через $P$,

b) плоскости, перпендикулярной $QBU$ и проходящей через $Q$ и $U$,

c) прямой, перпендикулярной $QBU$ и проходящей через $U$.

Вариант N13

Даны точки в пространстве
$C(0, -2, -9)$, $G(4, 3, -5)$, $K(0, -3, -9)$ и
$Q(4, -3, -13)$.

Найти угол между прямой $KC$ и плоскостью $z = 0$, угол между $KG$ и $GQ$, угол между плоскостями $KCG$ 
и $CGQ$.

Составить уравнение: 

a) плоскости, параллельной $KCG$ и проходящей через $Q$,

b) плоскости, перпендикулярной $KCG$ и проходящей через $K$ и $G$,

c) прямой, перпендикулярной $KCG$ и проходящей через $G$.

Вариант N14

Даны точки в пространстве
$G(-5, -5, -3)$, $H(-2, -4, -2)$, $R(-4, -10, -1)$ и
$V(-1, -3, 0)$.

Найти угол между прямой $GH$ и плоскостью $z = 0$, угол между $GV$ и $VR$, угол между плоскостями $GHV$ 
и $HVR$.

Составить уравнение: 

a) плоскости, параллельной $GHV$ и проходящей через $R$,

b) плоскости, перпендикулярной $GHV$ и проходящей через $G$ и $V$,

c) прямой, перпендикулярной $GHV$ и проходящей через $V$.

Вариант N15

Даны точки в пространстве
$C(-6, 2, -5)$, $F(-1, 9, -1)$, $U(-5, 5, -5)$ и
$W(6, -2, -13)$.

Найти угол между прямой $CU$ и плоскостью $z = 0$, угол между $CF$ и $FW$, угол между плоскостями $CUF$ 
и $UFW$.

Составить уравнение: 

a) плоскости, параллельной $CUF$ и проходящей через $W$,

b) плоскости, перпендикулярной $CUF$ и проходящей через $C$ и $F$,

c) прямой, перпендикулярной $CUF$ и проходящей через $F$.

Вариант N16

Даны точки в пространстве
$C(2, 10, 0)$, $G(-3, 5, -7)$, $P(1, 7, -3)$ и
$R(-9, -3, 3)$.

Найти угол между прямой $GP$ и плоскостью $z = 0$, угол между $GC$ и $CR$, угол между плоскостями $GPC$ 
и $PCR$.

Составить уравнение: 

a) плоскости, параллельной $GPC$ и проходящей через $R$,

b) плоскости, перпендикулярной $GPC$ и проходящей через $G$ и $C$,

c) прямой, перпендикулярной $GPC$ и проходящей через $C$.

Вариант N17

Даны точки в пространстве
$F(-5, 2, 9)$, $Q(-6, 2, 9)$, $V(-5, 4, 7)$ и
$W(-1, 4, 11)$.

Найти угол между прямой $FQ$ и плоскостью $z = 0$, угол между $FW$ и $WV$, угол между плоскостями $FQW$ 
и $QWV$.

Составить уравнение: 

a) плоскости, параллельной $FQW$ и проходящей через $V$,

b) плоскости, перпендикулярной $FQW$ и проходящей через $F$ и $W$,

c) прямой, перпендикулярной $FQW$ и проходящей через $W$.

Вариант N18

Даны точки в пространстве
$A(-9, -8, 15)$, $H(-3, -1, 2)$, $K(-1, 3, 4)$ и
$V(-7, -2, 1)$.

Найти угол между прямой $VH$ и плоскостью $z = 0$, угол между $VK$ и $KA$, угол между плоскостями $VHK$ 
и $HKA$.

Составить уравнение: 

a) плоскости, параллельной $VHK$ и проходящей через $A$,

b) плоскости, перпендикулярной $VHK$ и проходящей через $V$ и $K$,

c) прямой, перпендикулярной $VHK$ и проходящей через $K$.

Вариант N19

Даны точки в пространстве
$F(-4, 5, 13)$, $H(2, -19, 13)$, $M(-9, -2, 5)$ и
$W(-5, 2, 8)$.

Найти угол между прямой $MW$ и плоскостью $z = 0$, угол между $MF$ и $FH$, угол между плоскостями $MWF$ 
и $WFH$.

Составить уравнение: 

a) плоскости, параллельной $MWF$ и проходящей через $H$,

b) плоскости, перпендикулярной $MWF$ и проходящей через $M$ и $F$,

c) прямой, перпендикулярной $MWF$ и проходящей через $F$.

Вариант N20

Даны точки в пространстве
$C(11, -2, 11)$, $H(4, -7, 12)$, $M(5, -7, 9)$ и
$Q(8, -5, 10)$.

Найти угол между прямой $MQ$ и плоскостью $z = 0$, угол между $MC$ и $CH$, угол между плоскостями $MQC$ 
и $QCH$.

Составить уравнение: 

a) плоскости, параллельной $MQC$ и проходящей через $H$,

b) плоскости, перпендикулярной $MQC$ и проходящей через $M$ и $C$,

c) прямой, перпендикулярной $MQC$ и проходящей через $C$.

Вариант N21

Даны точки в пространстве
$C(3, -4, 7)$, $D(3, -3, 9)$, $H(7, -1, 10)$ и
$K(0, 4, 3)$.

Найти угол между прямой $CD$ и плоскостью $z = 0$, угол между $CH$ и $HK$, угол между плоскостями $CDH$ 
и $DHK$.

Составить уравнение: 

a) плоскости, параллельной $CDH$ и проходящей через $K$,

b) плоскости, перпендикулярной $CDH$ и проходящей через $C$ и $H$,

c) прямой, перпендикулярной $CDH$ и проходящей через $H$.

Вариант N22

Даны точки в пространстве
$B(-3, 5, -2)$, $D(-8, -2, -9)$, $M(-8, 1, -6)$ и
$U(-8, 13, -24)$.

Найти угол между прямой $DM$ и плоскостью $z = 0$, угол между $DB$ и $BU$, угол между плоскостями $DMB$ 
и $MBU$.

Составить уравнение: 

a) плоскости, параллельной $DMB$ и проходящей через $U$,

b) плоскости, перпендикулярной $DMB$ и проходящей через $D$ и $B$,

c) прямой, перпендикулярной $DMB$ и проходящей через $B$.

Вариант N23

Даны точки в пространстве
$A(5, 8, -2)$, $G(0, 1, -7)$, $Q(3, 4, -5)$ и
$R(1, -4, -1)$.

Найти угол между прямой $GQ$ и плоскостью $z = 0$, угол между $GA$ и $AR$, угол между плоскостями $GQA$ 
и $QAR$.

Составить уравнение: 

a) плоскости, параллельной $GQA$ и проходящей через $R$,

b) плоскости, перпендикулярной $GQA$ и проходящей через $G$ и $A$,

c) прямой, перпендикулярной $GQA$ и проходящей через $A$.

Вариант N24

Даны точки в пространстве
$D(-9, -8, -6)$, $F(-6, -6, -5)$, $K(-10, -9, -1)$ и
$W(-1, -1, -3)$.

Найти угол между прямой $DF$ и плоскостью $z = 0$, угол между $DW$ и $WK$, угол между плоскостями $DFW$ 
и $FWK$.

Составить уравнение: 

a) плоскости, параллельной $DFW$ и проходящей через $K$,

b) плоскости, перпендикулярной $DFW$ и проходящей через $D$ и $W$,

c) прямой, перпендикулярной $DFW$ и проходящей через $W$.

Вариант N25

Даны точки в пространстве
$B(-19, 11, 9)$, $G(-7, 2, 12)$, $Q(-7, 2, 9)$ и
$R(-4, 6, 17)$.

Найти угол между прямой $QG$ и плоскостью $z = 0$, угол между $QR$ и $RB$, угол между плоскостями $QGR$ 
и $GRB$.

Составить уравнение: 

a) плоскости, параллельной $QGR$ и проходящей через $B$,

b) плоскости, перпендикулярной $QGR$ и проходящей через $Q$ и $R$,

c) прямой, перпендикулярной $QGR$ и проходящей через $R$.

Вариант N26

Даны точки в пространстве
$A(-3, -8, 11)$, $D(-2, -7, 13)$, $U(-8, -11, 9)$ и
$W(-6, -9, 7)$.

Найти угол между прямой $WA$ и плоскостью $z = 0$, угол между $WD$ и $DU$, угол между плоскостями $WAD$ 
и $ADU$.

Составить уравнение: 

a) плоскости, параллельной $WAD$ и проходящей через $U$,

b) плоскости, перпендикулярной $WAD$ и проходящей через $W$ и $D$,

c) прямой, перпендикулярной $WAD$ и проходящей через $D$.

Вариант N27

Даны точки в пространстве
$F(3, 5, 10)$, $H(10, 4, -8)$, $K(-2, 0, 2)$ и
$Q(0, 4, 6)$.

Найти угол между прямой $KQ$ и плоскостью $z = 0$, угол между $KF$ и $FH$, угол между плоскостями $KQF$ 
и $QFH$.

Составить уравнение: 

a) плоскости, параллельной $KQF$ и проходящей через $H$,

b) плоскости, перпендикулярной $KQF$ и проходящей через $K$ и $F$,

c) прямой, перпендикулярной $KQF$ и проходящей через $F$.

Вариант N28

Даны точки в пространстве
$C(-2, 4, 8)$, $D(1, 6, 10)$, $P(-3, 4, 5)$ и
$U(-9, 11, 7)$.

Найти угол между прямой $PC$ и плоскостью $z = 0$, угол между $PD$ и $DU$, угол между плоскостями $PCD$ 
и $CDU$.

Составить уравнение: 

a) плоскости, параллельной $PCD$ и проходящей через $U$,

b) плоскости, перпендикулярной $PCD$ и проходящей через $P$ и $D$,

c) прямой, перпендикулярной $PCD$ и проходящей через $D$.

Вариант N29

Даны точки в пространстве
$A(-5, 3, 5)$, $C(0, 5, 10)$, $F(-4, 3, 5)$ и
$H(-4, 8, 3)$.

Найти угол между прямой $FA$ и плоскостью $z = 0$, угол между $FC$ и $CH$, угол между плоскостями $FAC$ 
и $ACH$.

Составить уравнение: 

a) плоскости, параллельной $FAC$ и проходящей через $H$,

b) плоскости, перпендикулярной $FAC$ и проходящей через $F$ и $C$,

c) прямой, перпендикулярной $FAC$ и проходящей через $C$.

Вариант N30

Даны точки в пространстве
$A(-3, -1, -6)$, $C(-7, -3, -6)$, $F(1, -19, -2)$ и
$K(1, 2, -2)$.

Найти угол между прямой $CA$ и плоскостью $z = 0$, угол между $CK$ и $KF$, угол между плоскостями $CAK$ 
и $AKF$.

Составить уравнение: 

a) плоскости, параллельной $CAK$ и проходящей через $F$,

b) плоскости, перпендикулярной $CAK$ и проходящей через $C$ и $K$,

c) прямой, перпендикулярной $CAK$ и проходящей через $K$.

Вариант N31

Даны точки в пространстве
$A(6, 3, 13)$, $F(4, 0, 9)$, $H(10, -16, 11)$ и
$U(0, -4, 7)$.

Найти угол между прямой $UF$ и плоскостью $z = 0$, угол между $UA$ и $AH$, угол между плоскостями $UFA$ 
и $FAH$.

Составить уравнение: 

a) плоскости, параллельной $UFA$ и проходящей через $H$,

b) плоскости, перпендикулярной $UFA$ и проходящей через $U$ и $A$,

c) прямой, перпендикулярной $UFA$ и проходящей через $A$.

Вариант N32

Даны точки в пространстве
$F(2, 0, -3)$, $H(-6, -5, -6)$, $M(-3, -5, -4)$ и
$U(-16, 2, 9)$.

Найти угол между прямой $HM$ и плоскостью $z = 0$, угол между $HF$ и $FU$, угол между плоскостями $HMF$ 
и $MFU$.

Составить уравнение: 

a) плоскости, параллельной $HMF$ и проходящей через $U$,

b) плоскости, перпендикулярной $HMF$ и проходящей через $H$ и $F$,

c) прямой, перпендикулярной $HMF$ и проходящей через $F$.

Вариант N33

Даны точки в пространстве
$A(8, -6, -8)$, $Q(7, -2, -8)$, $V(15, 0, 1)$ и
$W(11, -1, -4)$.

Найти угол между прямой $QW$ и плоскостью $z = 0$, угол между $QV$ и $VA$, угол между плоскостями $QWV$ 
и $WVA$.

Составить уравнение: 

a) плоскости, параллельной $QWV$ и проходящей через $A$,

b) плоскости, перпендикулярной $QWV$ и проходящей через $Q$ и $V$,

c) прямой, перпендикулярной $QWV$ и проходящей через $V$.

Вариант N34

Даны точки в пространстве
$A(-6, 4, 4)$, $C(0, 7, 11)$, $H(-4, 5, 8)$ и
$Q(-11, 14, 4)$.

Найти угол между прямой $AH$ и плоскостью $z = 0$, угол между $AC$ и $CQ$, угол между плоскостями $AHC$ 
и $HCQ$.

Составить уравнение: 

a) плоскости, параллельной $AHC$ и проходящей через $Q$,

b) плоскости, перпендикулярной $AHC$ и проходящей через $A$ и $C$,

c) прямой, перпендикулярной $AHC$ и проходящей через $C$.

Вариант N35

Даны точки в пространстве
$A(-1, -3, 5)$, $B(0, -3, 4)$, $D(4, 1, 10)$ и
$G(0, -2, 6)$.

Найти угол между прямой $AG$ и плоскостью $z = 0$, угол между $AD$ и $DB$, угол между плоскостями $AGD$ 
и $GDB$.

Составить уравнение: 

a) плоскости, параллельной $AGD$ и проходящей через $B$,

b) плоскости, перпендикулярной $AGD$ и проходящей через $A$ и $D$,

c) прямой, перпендикулярной $AGD$ и проходящей через $D$.

Вариант N36

Даны точки в пространстве
$A(3, 10, -3)$, $B(-4, 8, -8)$, $C(-1, 9, -7)$ и
$R(-1, 0, -9)$.

Найти угол между прямой $BC$ и плоскостью $z = 0$, угол между $BA$ и $AR$, угол между плоскостями $BCA$ 
и $CAR$.

Составить уравнение: 

a) плоскости, параллельной $BCA$ и проходящей через $R$,

b) плоскости, перпендикулярной $BCA$ и проходящей через $B$ и $A$,

c) прямой, перпендикулярной $BCA$ и проходящей через $A$.

Вариант N37

Даны точки в пространстве
$G(3, 4, 4)$, $K(2, 11, -11)$, $Q(6, 5, 5)$ и
$V(2, 0, 0)$.

Найти угол между прямой $VG$ и плоскостью $z = 0$, угол между $VQ$ и $QK$, угол между плоскостями $VGQ$ 
и $GQK$.

Составить уравнение: 

a) плоскости, параллельной $VGQ$ и проходящей через $K$,

b) плоскости, перпендикулярной $VGQ$ и проходящей через $V$ и $Q$,

c) прямой, перпендикулярной $VGQ$ и проходящей через $Q$.

Вариант N38

Даны точки в пространстве
$A(-4, 5, 11)$, $B(-8, 5, 8)$, $G(0, 9, 13)$ и
$R(-20, 9, 24)$.

Найти угол между прямой $BA$ и плоскостью $z = 0$, угол между $BG$ и $GR$, угол между плоскостями $BAG$ 
и $AGR$.

Составить уравнение: 

a) плоскости, параллельной $BAG$ и проходящей через $R$,

b) плоскости, перпендикулярной $BAG$ и проходящей через $B$ и $G$,

c) прямой, перпендикулярной $BAG$ и проходящей через $G$.

Вариант N39

Даны точки в пространстве
$B(11, 4, 6)$, $D(3, -2, -1)$, $R(6, -1, 2)$ и
$W(-8, 1, 9)$.

Найти угол между прямой $DR$ и плоскостью $z = 0$, угол между $DB$ и $BW$, угол между плоскостями $DRB$ 
и $RBW$.

Составить уравнение: 

a) плоскости, параллельной $DRB$ и проходящей через $W$,

b) плоскости, перпендикулярной $DRB$ и проходящей через $D$ и $B$,

c) прямой, перпендикулярной $DRB$ и проходящей через $B$.

Вариант N40

Даны точки в пространстве
$A(-2, -2, 5)$, $M(2, 2, 10)$, $R(-16, -2, 9)$ и
$W(-5, -3, 1)$.

Найти угол между прямой $WA$ и плоскостью $z = 0$, угол между $WM$ и $MR$, угол между плоскостями $WAM$ 
и $AMR$.

Составить уравнение: 

a) плоскости, параллельной $WAM$ и проходящей через $R$,

b) плоскости, перпендикулярной $WAM$ и проходящей через $W$ и $M$,

c) прямой, перпендикулярной $WAM$ и проходящей через $M$.

Вариант N41

Даны точки в пространстве
$B(0, 4, 11)$, $F(-1, 3, 9)$, $G(-3, 2, 7)$ и
$M(-3, 0, 8)$.

Найти угол между прямой $GF$ и плоскостью $z = 0$, угол между $GB$ и $BM$, угол между плоскостями $GFB$ 
и $FBM$.

Составить уравнение: 

a) плоскости, параллельной $GFB$ и проходящей через $M$,

b) плоскости, перпендикулярной $GFB$ и проходящей через $G$ и $B$,

c) прямой, перпендикулярной $GFB$ и проходящей через $B$.

Вариант N42

Даны точки в пространстве
$G(13, 0, -1)$, $K(9, -2, -5)$, $Q(9, 5, -11)$ и
$U(9, -3, -7)$.

Найти угол между прямой $UK$ и плоскостью $z = 0$, угол между $UG$ и $GQ$, угол между плоскостями $UKG$ 
и $KGQ$.

Составить уравнение: 

a) плоскости, параллельной $UKG$ и проходящей через $Q$,

b) плоскости, перпендикулярной $UKG$ и проходящей через $U$ и $G$,

c) прямой, перпендикулярной $UKG$ и проходящей через $G$.

Вариант N43

Даны точки в пространстве
$G(-2, 8, 12)$, $P(-6, 3, 7)$, $Q(-5, 3, 10)$ и
$R(-21, 10, 12)$.

Найти угол между прямой $PQ$ и плоскостью $z = 0$, угол между $PG$ и $GR$, угол между плоскостями $PQG$ 
и $QGR$.

Составить уравнение: 

a) плоскости, параллельной $PQG$ и проходящей через $R$,

b) плоскости, перпендикулярной $PQG$ и проходящей через $P$ и $G$,

c) прямой, перпендикулярной $PQG$ и проходящей через $G$.

Вариант N44

Даны точки в пространстве
$B(12, 10, 12)$, $F(5, 5, 5)$, $M(11, 12, -6)$ и
$Q(7, 8, 8)$.

Найти угол между прямой $FQ$ и плоскостью $z = 0$, угол между $FB$ и $BM$, угол между плоскостями $FQB$ 
и $QBM$.

Составить уравнение: 

a) плоскости, параллельной $FQB$ и проходящей через $M$,

b) плоскости, перпендикулярной $FQB$ и проходящей через $F$ и $B$,

c) прямой, перпендикулярной $FQB$ и проходящей через $B$.

Вариант N45

Даны точки в пространстве
$B(2, -6, -3)$, $K(6, -2, 0)$, $M(3, -4, -7)$ и
$Q(4, -3, -1)$.

Найти угол между прямой $BQ$ и плоскостью $z = 0$, угол между $BK$ и $KM$, угол между плоскостями $BQK$ 
и $QKM$.

Составить уравнение: 

a) плоскости, параллельной $BQK$ и проходящей через $M$,

b) плоскости, перпендикулярной $BQK$ и проходящей через $B$ и $K$,

c) прямой, перпендикулярной $BQK$ и проходящей через $K$.

Вариант N46

Даны точки в пространстве
$F(2, -5, 5)$, $K(-18, -17, 25)$, $P(6, -5, 9)$ и
$V(8, 0, 14)$.

Найти угол между прямой $FP$ и плоскостью $z = 0$, угол между $FV$ и $VK$, угол между плоскостями $FPV$ 
и $PVK$.

Составить уравнение: 

a) плоскости, параллельной $FPV$ и проходящей через $K$,

b) плоскости, перпендикулярной $FPV$ и проходящей через $F$ и $V$,

c) прямой, перпендикулярной $FPV$ и проходящей через $V$.

Вариант N47

Даны точки в пространстве
$F(7, -3, -1)$, $M(15, 5, -9)$, $R(11, 0, 6)$ и
$V(7, -1, 1)$.

Найти угол между прямой $FV$ и плоскостью $z = 0$, угол между $FR$ и $RM$, угол между плоскостями $FVR$ 
и $VRM$.

Составить уравнение: 

a) плоскости, параллельной $FVR$ и проходящей через $M$,

b) плоскости, перпендикулярной $FVR$ и проходящей через $F$ и $R$,

c) прямой, перпендикулярной $FVR$ и проходящей через $R$.

Вариант N48

Даны точки в пространстве
$B(10, -6, -6)$, $K(8, -9, -9)$, $V(9, -9, -9)$ и
$W(9, -6, -12)$.

Найти угол между прямой $VK$ и плоскостью $z = 0$, угол между $VB$ и $BW$, угол между плоскостями $VKB$ 
и $KBW$.

Составить уравнение: 

a) плоскости, параллельной $VKB$ и проходящей через $W$,

b) плоскости, перпендикулярной $VKB$ и проходящей через $V$ и $B$,

c) прямой, перпендикулярной $VKB$ и проходящей через $B$.

Вариант N49

Даны точки в пространстве
$G(18, -8, -10)$, $H(0, 6, -8)$, $M(1, 7, -3)$ и
$V(-2, 2, -8)$.

Найти угол между прямой $VH$ и плоскостью $z = 0$, угол между $VM$ и $MG$, угол между плоскостями $VHM$ 
и $HMG$.

Составить уравнение: 

a) плоскости, параллельной $VHM$ и проходящей через $G$,

b) плоскости, перпендикулярной $VHM$ и проходящей через $V$ и $M$,

c) прямой, перпендикулярной $VHM$ и проходящей через $M$.

Вариант N50

Даны точки в пространстве
$D(2, -4, -7)$, $F(4, -2, -10)$, $H(3, -2, -5)$ и
$Q(6, 1, -1)$.

Найти угол между прямой $DH$ и плоскостью $z = 0$, угол между $DQ$ и $QF$, угол между плоскостями $DHQ$ 
и $HQF$.

Составить уравнение: 

a) плоскости, параллельной $DHQ$ и проходящей через $F$,

b) плоскости, перпендикулярной $DHQ$ и проходящей через $D$ и $Q$,

c) прямой, перпендикулярной $DHQ$ и проходящей через $Q$.

Вариант N51

Даны точки в пространстве
$F(-3, 1, 0)$, $P(-2, 5, 4)$, $Q(-5, 2, 3)$ и
$U(0, 6, 5)$.

Найти угол между прямой $QP$ и плоскостью $z = 0$, угол между $QU$ и $UF$, угол между плоскостями $QPU$ 
и $PUF$.

Составить уравнение: 

a) плоскости, параллельной $QPU$ и проходящей через $F$,

b) плоскости, перпендикулярной $QPU$ и проходящей через $Q$ и $U$,

c) прямой, перпендикулярной $QPU$ и проходящей через $U$.

Вариант N52

Даны точки в пространстве
$F(6, -1, -5)$, $H(10, 2, -2)$, $M(3, -5, 3)$ и
$P(14, 7, 2)$.

Найти угол между прямой $FH$ и плоскостью $z = 0$, угол между $FP$ и $PM$, угол между плоскостями $FHP$ 
и $HPM$.

Составить уравнение: 

a) плоскости, параллельной $FHP$ и проходящей через $M$,

b) плоскости, перпендикулярной $FHP$ и проходящей через $F$ и $P$,

c) прямой, перпендикулярной $FHP$ и проходящей через $P$.

Вариант N53

Даны точки в пространстве
$A(3, -7, -6)$, $B(6, 1, -3)$, $K(-4, -4, -7)$ и
$R(4, -4, -4)$.

Найти угол между прямой $AR$ и плоскостью $z = 0$, угол между $AB$ и $BK$, угол между плоскостями $ARB$ 
и $RBK$.

Составить уравнение: 

a) плоскости, параллельной $ARB$ и проходящей через $K$,

b) плоскости, перпендикулярной $ARB$ и проходящей через $A$ и $B$,

c) прямой, перпендикулярной $ARB$ и проходящей через $B$.

Вариант N54

Даны точки в пространстве
$G(0, 13, 8)$, $H(-4, 5, 2)$, $P(-3, 9, 5)$ и
$U(-4, 11, -6)$.

Найти угол между прямой $HP$ и плоскостью $z = 0$, угол между $HG$ и $GU$, угол между плоскостями $HPG$ 
и $PGU$.

Составить уравнение: 

a) плоскости, параллельной $HPG$ и проходящей через $U$,

b) плоскости, перпендикулярной $HPG$ и проходящей через $H$ и $G$,

c) прямой, перпендикулярной $HPG$ и проходящей через $G$.

Вариант N55

Даны точки в пространстве
$B(10, 1, 8)$, $D(7, -4, 4)$, $H(6, -13, 8)$ и
$P(4, -7, 2)$.

Найти угол между прямой $PD$ и плоскостью $z = 0$, угол между $PB$ и $BH$, угол между плоскостями $PDB$ 
и $DBH$.

Составить уравнение: 

a) плоскости, параллельной $PDB$ и проходящей через $H$,

b) плоскости, перпендикулярной $PDB$ и проходящей через $P$ и $B$,

c) прямой, перпендикулярной $PDB$ и проходящей через $B$.

Вариант N56

Даны точки в пространстве
$F(13, -6, 8)$, $M(16, 6, 14)$, $P(9, 1, 7)$ и
$W(12, 3, 9)$.

Найти угол между прямой $PW$ и плоскостью $z = 0$, угол между $PM$ и $MF$, угол между плоскостями $PWM$ 
и $WMF$.

Составить уравнение: 

a) плоскости, параллельной $PWM$ и проходящей через $F$,

b) плоскости, перпендикулярной $PWM$ и проходящей через $P$ и $M$,

c) прямой, перпендикулярной $PWM$ и проходящей через $M$.

Вариант N57

Даны точки в пространстве
$B(10, 11, -16)$, $C(12, 11, 0)$, $Q(8, 7, -5)$ и
$R(7, 4, -8)$.

Найти угол между прямой $RQ$ и плоскостью $z = 0$, угол между $RC$ и $CB$, угол между плоскостями $RQC$ 
и $QCB$.

Составить уравнение: 

a) плоскости, параллельной $RQC$ и проходящей через $B$,

b) плоскости, перпендикулярной $RQC$ и проходящей через $R$ и $C$,

c) прямой, перпендикулярной $RQC$ и проходящей через $C$.

Вариант N58

Даны точки в пространстве
$C(7, 3, 3)$, $F(7, 7, 4)$, $R(12, 8, -17)$ и
$U(12, 10, 6)$.

Найти угол между прямой $CF$ и плоскостью $z = 0$, угол между $CU$ и $UR$, угол между плоскостями $CFU$ 
и $FUR$.

Составить уравнение: 

a) плоскости, параллельной $CFU$ и проходящей через $R$,

b) плоскости, перпендикулярной $CFU$ и проходящей через $C$ и $U$,

c) прямой, перпендикулярной $CFU$ и проходящей через $U$.

Вариант N59

Даны точки в пространстве
$D(-1, 6, -9)$, $H(-4, 11, -12)$, $U(4, 12, -4)$ и
$V(1, 9, -6)$.

Найти угол между прямой $DV$ и плоскостью $z = 0$, угол между $DU$ и $UH$, угол между плоскостями $DVU$ 
и $VUH$.

Составить уравнение: 

a) плоскости, параллельной $DVU$ и проходящей через $H$,

b) плоскости, перпендикулярной $DVU$ и проходящей через $D$ и $U$,

c) прямой, перпендикулярной $DVU$ и проходящей через $U$.

Вариант N60

Даны точки в пространстве
$F(6, 5, 0)$, $G(2, -3, -5)$, $K(3, 0, -5)$ и
$P(17, -8, -9)$.

Найти угол между прямой $GK$ и плоскостью $z = 0$, угол между $GF$ и $FP$, угол между плоскостями $GKF$ 
и $KFP$.

Составить уравнение: 

a) плоскости, параллельной $GKF$ и проходящей через $P$,

b) плоскости, перпендикулярной $GKF$ и проходящей через $G$ и $F$,

c) прямой, перпендикулярной $GKF$ и проходящей через $F$.

Вариант N61

Даны точки в пространстве
$C(-3, 1, 6)$, $K(-1, 4, 6)$, $M(9, -7, 2)$ и
$Q(3, 8, 10)$.

Найти угол между прямой $CK$ и плоскостью $z = 0$, угол между $CQ$ и $QM$, угол между плоскостями $CKQ$ 
и $KQM$.

Составить уравнение: 

a) плоскости, параллельной $CKQ$ и проходящей через $M$,

b) плоскости, перпендикулярной $CKQ$ и проходящей через $C$ и $Q$,

c) прямой, перпендикулярной $CKQ$ и проходящей через $Q$.

Вариант N62

Даны точки в пространстве
$B(13, -8, -9)$, $G(3, 14, -2)$, $P(1, 12, -7)$ и
$U(-3, 8, -9)$.

Найти угол между прямой $UP$ и плоскостью $z = 0$, угол между $UG$ и $GB$, угол между плоскостями $UPG$ 
и $PGB$.

Составить уравнение: 

a) плоскости, параллельной $UPG$ и проходящей через $B$,

b) плоскости, перпендикулярной $UPG$ и проходящей через $U$ и $G$,

c) прямой, перпендикулярной $UPG$ и проходящей через $G$.

Вариант N63

Даны точки в пространстве
$A(9, 5, -7)$, $D(13, 7, -4)$, $G(8, -8, -3)$ и
$Q(5, 4, -7)$.

Найти угол между прямой $QA$ и плоскостью $z = 0$, угол между $QD$ и $DG$, угол между плоскостями $QAD$ 
и $ADG$.

Составить уравнение: 

a) плоскости, параллельной $QAD$ и проходящей через $G$,

b) плоскости, перпендикулярной $QAD$ и проходящей через $Q$ и $D$,

c) прямой, перпендикулярной $QAD$ и проходящей через $D$.

Вариант N64

Даны точки в пространстве
$A(11, 12, 12)$, $Q(4, -11, 18)$, $V(10, 8, 7)$ и
$W(6, 6, 4)$.

Найти угол между прямой $WV$ и плоскостью $z = 0$, угол между $WA$ и $AQ$, угол между плоскостями $WVA$ 
и $VAQ$.

Составить уравнение: 

a) плоскости, параллельной $WVA$ и проходящей через $Q$,

b) плоскости, перпендикулярной $WVA$ и проходящей через $W$ и $A$,

c) прямой, перпендикулярной $WVA$ и проходящей через $A$.

Вариант N65

Даны точки в пространстве
$F(11, 2, -2)$, $H(7, -4, -6)$, $Q(10, -1, -3)$ и
$U(9, -5, -5)$.

Найти угол между прямой $UQ$ и плоскостью $z = 0$, угол между $UF$ и $FH$, угол между плоскостями $UQF$ 
и $QFH$.

Составить уравнение: 

a) плоскости, параллельной $UQF$ и проходящей через $H$,

b) плоскости, перпендикулярной $UQF$ и проходящей через $U$ и $F$,

c) прямой, перпендикулярной $UQF$ и проходящей через $F$.

Вариант N66

Даны точки в пространстве
$C(-6, 20, -13)$, $G(9, 5, 1)$, $U(8, 2, -3)$ и
$W(14, 10, 3)$.

Найти угол между прямой $UG$ и плоскостью $z = 0$, угол между $UW$ и $WC$, угол между плоскостями $UGW$ 
и $GWC$.

Составить уравнение: 

a) плоскости, параллельной $UGW$ и проходящей через $C$,

b) плоскости, перпендикулярной $UGW$ и проходящей через $U$ и $W$,

c) прямой, перпендикулярной $UGW$ и проходящей через $W$.

Вариант N67

Даны точки в пространстве
$C(3, -2, 14)$, $Q(-1, -7, 9)$, $V(2, -18, 15)$ и
$W(-3, -8, 9)$.

Найти угол между прямой $WQ$ и плоскостью $z = 0$, угол между $WC$ и $CV$, угол между плоскостями $WQC$ 
и $QCV$.

Составить уравнение: 

a) плоскости, параллельной $WQC$ и проходящей через $V$,

b) плоскости, перпендикулярной $WQC$ и проходящей через $W$ и $C$,

c) прямой, перпендикулярной $WQC$ и проходящей через $C$.

Вариант N68

Даны точки в пространстве
$B(-5, -5, 8)$, $F(-1, 2, 10)$, $G(3, -5, -8)$ и
$U(-5, -1, 8)$.

Найти угол между прямой $BU$ и плоскостью $z = 0$, угол между $BF$ и $FG$, угол между плоскостями $BUF$ 
и $UFG$.

Составить уравнение: 

a) плоскости, параллельной $BUF$ и проходящей через $G$,

b) плоскости, перпендикулярной $BUF$ и проходящей через $B$ и $F$,

c) прямой, перпендикулярной $BUF$ и проходящей через $F$.

Вариант N69

Даны точки в пространстве
$D(1, 0, -2)$, $G(0, -1, 2)$, $H(3, 0, 6)$ и
$V(-1, -2, 0)$.

Найти угол между прямой $VG$ и плоскостью $z = 0$, угол между $VH$ и $HD$, угол между плоскостями $VGH$ 
и $GHD$.

Составить уравнение: 

a) плоскости, параллельной $VGH$ и проходящей через $D$,

b) плоскости, перпендикулярной $VGH$ и проходящей через $V$ и $H$,

c) прямой, перпендикулярной $VGH$ и проходящей через $H$.

Вариант N70

Даны точки в пространстве
$A(0, 6, 0)$, $D(-3, 5, -1)$, $K(-7, 1, -5)$ и
$Q(-7, 9, -13)$.

Найти угол между прямой $KD$ и плоскостью $z = 0$, угол между $KA$ и $AQ$, угол между плоскостями $KDA$ 
и $DAQ$.

Составить уравнение: 

a) плоскости, параллельной $KDA$ и проходящей через $Q$,

b) плоскости, перпендикулярной $KDA$ и проходящей через $K$ и $A$,

c) прямой, перпендикулярной $KDA$ и проходящей через $A$.

Вариант N71

Даны точки в пространстве
$G(-1, 10, 10)$, $M(-5, -6, 14)$, $P(-9, 2, 6)$ и
$U(-5, 5, 7)$.

Найти угол между прямой $PU$ и плоскостью $z = 0$, угол между $PG$ и $GM$, угол между плоскостями $PUG$ 
и $UGM$.

Составить уравнение: 

a) плоскости, параллельной $PUG$ и проходящей через $M$,

b) плоскости, перпендикулярной $PUG$ и проходящей через $P$ и $G$,

c) прямой, перпендикулярной $PUG$ и проходящей через $G$.

Вариант N72

Даны точки в пространстве
$F(10, 3, 7)$, $G(1, -2, 2)$, $H(21, -22, -14)$ и
$R(5, 2, 2)$.

Найти угол между прямой $GR$ и плоскостью $z = 0$, угол между $GF$ и $FH$, угол между плоскостями $GRF$ 
и $RFH$.

Составить уравнение: 

a) плоскости, параллельной $GRF$ и проходящей через $H$,

b) плоскости, перпендикулярной $GRF$ и проходящей через $G$ и $F$,

c) прямой, перпендикулярной $GRF$ и проходящей через $F$.

Вариант N73

Даны точки в пространстве
$D(11, 13, 12)$, $F(7, 7, 10)$, $H(6, 9, 9)$ и
$W(10, 12, 11)$.

Найти угол между прямой $HW$ и плоскостью $z = 0$, угол между $HD$ и $DF$, угол между плоскостями $HWD$ 
и $WDF$.

Составить уравнение: 

a) плоскости, параллельной $HWD$ и проходящей через $F$,

b) плоскости, перпендикулярной $HWD$ и проходящей через $H$ и $D$,

c) прямой, перпендикулярной $HWD$ и проходящей через $D$.

Вариант N74

Даны точки в пространстве
$B(9, -1, -2)$, $C(11, 1, 2)$, $G(15, 2, 6)$ и
$U(13, 7, -8)$.

Найти угол между прямой $BC$ и плоскостью $z = 0$, угол между $BG$ и $GU$, угол между плоскостями $BCG$ 
и $CGU$.

Составить уравнение: 

a) плоскости, параллельной $BCG$ и проходящей через $U$,

b) плоскости, перпендикулярной $BCG$ и проходящей через $B$ и $G$,

c) прямой, перпендикулярной $BCG$ и проходящей через $G$.

Вариант N75

Даны точки в пространстве
$A(-9, 6, 7)$, $H(-5, 5, -5)$, $U(2, 9, -3)$ и
$W(-2, 5, -4)$.

Найти угол между прямой $HW$ и плоскостью $z = 0$, угол между $HU$ и $UA$, угол между плоскостями $HWU$ 
и $WUA$.

Составить уравнение: 

a) плоскости, параллельной $HWU$ и проходящей через $A$,

b) плоскости, перпендикулярной $HWU$ и проходящей через $H$ и $U$,

c) прямой, перпендикулярной $HWU$ и проходящей через $U$.

Вариант N76

Даны точки в пространстве
$A(-1, 5, 3)$, $H(4, -7, 12)$, $Q(2, 7, 4)$ и
$U(5, 12, 9)$.

Найти угол между прямой $AQ$ и плоскостью $z = 0$, угол между $AU$ и $UH$, угол между плоскостями $AQU$ 
и $QUH$.

Составить уравнение: 

a) плоскости, параллельной $AQU$ и проходящей через $H$,

b) плоскости, перпендикулярной $AQU$ и проходящей через $A$ и $U$,

c) прямой, перпендикулярной $AQU$ и проходящей через $U$.

Вариант N77

Даны точки в пространстве
$A(-23, 14, 10)$, $F(-6, 2, 2)$, $H(-2, 5, 6)$ и
$Q(2, 10, 7)$.

Найти угол между прямой $FH$ и плоскостью $z = 0$, угол между $FQ$ и $QA$, угол между плоскостями $FHQ$ 
и $HQA$.

Составить уравнение: 

a) плоскости, параллельной $FHQ$ и проходящей через $A$,

b) плоскости, перпендикулярной $FHQ$ и проходящей через $F$ и $Q$,

c) прямой, перпендикулярной $FHQ$ и проходящей через $Q$.

Вариант N78

Даны точки в пространстве
$A(13, 1, -2)$, $F(15, -8, -16)$, $P(8, -2, -6)$ и
$U(6, -5, -7)$.

Найти угол между прямой $UP$ и плоскостью $z = 0$, угол между $UA$ и $AF$, угол между плоскостями $UPA$ 
и $PAF$.

Составить уравнение: 

a) плоскости, параллельной $UPA$ и проходящей через $F$,

b) плоскости, перпендикулярной $UPA$ и проходящей через $U$ и $A$,

c) прямой, перпендикулярной $UPA$ и проходящей через $A$.

Вариант N79

Даны точки в пространстве
$C(-9, -2, -5)$, $F(-2, 2, 2)$, $M(-4, -2, -10)$ и
$R(-5, 1, -1)$.

Найти угол между прямой $CR$ и плоскостью $z = 0$, угол между $CF$ и $FM$, угол между плоскостями $CRF$ 
и $RFM$.

Составить уравнение: 

a) плоскости, параллельной $CRF$ и проходящей через $M$,

b) плоскости, перпендикулярной $CRF$ и проходящей через $C$ и $F$,

c) прямой, перпендикулярной $CRF$ и проходящей через $F$.

Вариант N80

Даны точки в пространстве
$A(13, 5, 3)$, $D(16, -4, -4)$, $G(9, 2, 1)$ и
$H(16, 6, 6)$.

Найти угол между прямой $GA$ и плоскостью $z = 0$, угол между $GH$ и $HD$, угол между плоскостями $GAH$ 
и $AHD$.

Составить уравнение: 

a) плоскости, параллельной $GAH$ и проходящей через $D$,

b) плоскости, перпендикулярной $GAH$ и проходящей через $G$ и $H$,

c) прямой, перпендикулярной $GAH$ и проходящей через $H$.

Вариант N81

Даны точки в пространстве
$B(-8, 4, -2)$, $H(-6, 10, 1)$, $M(-5, 4, -4)$ и
$Q(-8, 5, -2)$.

Найти угол между прямой $BQ$ и плоскостью $z = 0$, угол между $BH$ и $HM$, угол между плоскостями $BQH$ 
и $QHM$.

Составить уравнение: 

a) плоскости, параллельной $BQH$ и проходящей через $M$,

b) плоскости, перпендикулярной $BQH$ и проходящей через $B$ и $H$,

c) прямой, перпендикулярной $BQH$ и проходящей через $H$.

Вариант N82

Даны точки в пространстве
$A(-4, 15, 7)$, $G(9, 6, 12)$, $U(5, 3, 10)$ и
$V(5, 3, 7)$.

Найти угол между прямой $VU$ и плоскостью $z = 0$, угол между $VG$ и $GA$, угол между плоскостями $VUG$ 
и $UGA$.

Составить уравнение: 

a) плоскости, параллельной $VUG$ и проходящей через $A$,

b) плоскости, перпендикулярной $VUG$ и проходящей через $V$ и $G$,

c) прямой, перпендикулярной $VUG$ и проходящей через $G$.

Вариант N83

Даны точки в пространстве
$K(13, 11, 2)$, $P(8, 6, -4)$, $R(-6, 20, -4)$ и
$V(9, 7, 0)$.

Найти угол между прямой $PV$ и плоскостью $z = 0$, угол между $PK$ и $KR$, угол между плоскостями $PVK$ 
и $VKR$.

Составить уравнение: 

a) плоскости, параллельной $PVK$ и проходящей через $R$,

b) плоскости, перпендикулярной $PVK$ и проходящей через $P$ и $K$,

c) прямой, перпендикулярной $PVK$ и проходящей через $K$.

Вариант N84

Даны точки в пространстве
$F(0, -4, 1)$, $H(6, -5, 8)$, $K(4, -9, 5)$ и
$P(9, -1, 10)$.

Найти угол между прямой $KH$ и плоскостью $z = 0$, угол между $KP$ и $PF$, угол между плоскостями $KHP$ 
и $HPF$.

Составить уравнение: 

a) плоскости, параллельной $KHP$ и проходящей через $F$,

b) плоскости, перпендикулярной $KHP$ и проходящей через $K$ и $P$,

c) прямой, перпендикулярной $KHP$ и проходящей через $P$.

Вариант N85

Даны точки в пространстве
$C(-1, 12, 9)$, $D(8, 6, 9)$, $G(8, 6, 12)$ и
$K(10, 9, 16)$.

Найти угол между прямой $DG$ и плоскостью $z = 0$, угол между $DK$ и $KC$, угол между плоскостями $DGK$ 
и $GKC$.

Составить уравнение: 

a) плоскости, параллельной $DGK$ и проходящей через $C$,

b) плоскости, перпендикулярной $DGK$ и проходящей через $D$ и $K$,

c) прямой, перпендикулярной $DGK$ и проходящей через $K$.

Вариант N86

Даны точки в пространстве
$G(4, 0, 12)$, $H(3, -1, 8)$, $M(-9, 7, 9)$ и
$R(7, 4, 16)$.

Найти угол между прямой $HG$ и плоскостью $z = 0$, угол между $HR$ и $RM$, угол между плоскостями $HGR$ 
и $GRM$.

Составить уравнение: 

a) плоскости, параллельной $HGR$ и проходящей через $M$,

b) плоскости, перпендикулярной $HGR$ и проходящей через $H$ и $R$,

c) прямой, перпендикулярной $HGR$ и проходящей через $R$.

Вариант N87

Даны точки в пространстве
$B(1, 1, -1)$, $M(10, -6, -2)$, $P(0, -3, -3)$ и
$V(2, 6, 4)$.

Найти угол между прямой $PB$ и плоскостью $z = 0$, угол между $PV$ и $VM$, угол между плоскостями $PBV$ 
и $BVM$.

Составить уравнение: 

a) плоскости, параллельной $PBV$ и проходящей через $M$,

b) плоскости, перпендикулярной $PBV$ и проходящей через $P$ и $V$,

c) прямой, перпендикулярной $PBV$ и проходящей через $V$.

Вариант N88

Даны точки в пространстве
$C(10, 14, 1)$, $K(7, 11, -2)$, $Q(6, 8, -3)$ и
$U(12, 8, -9)$.

Найти угол между прямой $QK$ и плоскостью $z = 0$, угол между $QC$ и $CU$, угол между плоскостями $QKC$ 
и $KCU$.

Составить уравнение: 

a) плоскости, параллельной $QKC$ и проходящей через $U$,

b) плоскости, перпендикулярной $QKC$ и проходящей через $Q$ и $C$,

c) прямой, перпендикулярной $QKC$ и проходящей через $C$.

Вариант N89

Даны точки в пространстве
$A(13, 11, 5)$, $M(9, 6, 0)$, $P(-11, 22, -4)$ и
$R(9, 6, -4)$.

Найти угол между прямой $RM$ и плоскостью $z = 0$, угол между $RA$ и $AP$, угол между плоскостями $RMA$ 
и $MAP$.

Составить уравнение: 

a) плоскости, параллельной $RMA$ и проходящей через $P$,

b) плоскости, перпендикулярной $RMA$ и проходящей через $R$ и $A$,

c) прямой, перпендикулярной $RMA$ и проходящей через $A$.

Вариант N90

Даны точки в пространстве
$C(11, 8, 3)$, $M(9, 6, 0)$, $U(17, 0, -4)$ и
$W(8, 3, 0)$.

Найти угол между прямой $WM$ и плоскостью $z = 0$, угол между $WC$ и $CU$, угол между плоскостями $WMC$ 
и $MCU$.

Составить уравнение: 

a) плоскости, параллельной $WMC$ и проходящей через $U$,

b) плоскости, перпендикулярной $WMC$ и проходящей через $W$ и $C$,

c) прямой, перпендикулярной $WMC$ и проходящей через $C$.

Вариант N91

Даны точки в пространстве
$K(14, 13, -1)$, $P(8, 6, -3)$, $R(9, 10, -20)$ и
$V(9, 10, -2)$.

Найти угол между прямой $PV$ и плоскостью $z = 0$, угол между $PK$ и $KR$, угол между плоскостями $PVK$ 
и $VKR$.

Составить уравнение: 

a) плоскости, параллельной $PVK$ и проходящей через $R$,

b) плоскости, перпендикулярной $PVK$ и проходящей через $P$ и $K$,

c) прямой, перпендикулярной $PVK$ и проходящей через $K$.

Вариант N92

Даны точки в пространстве
$G(3, -1, 9)$, $H(9, 5, 3)$, $P(3, 1, 11)$ и
$Q(6, 3, 16)$.

Найти угол между прямой $GP$ и плоскостью $z = 0$, угол между $GQ$ и $QH$, угол между плоскостями $GPQ$ 
и $PQH$.

Составить уравнение: 

a) плоскости, параллельной $GPQ$ и проходящей через $H$,

b) плоскости, перпендикулярной $GPQ$ и проходящей через $G$ и $Q$,

c) прямой, перпендикулярной $GPQ$ и проходящей через $Q$.

Вариант N93

Даны точки в пространстве
$B(7, -5, 6)$, $K(11, 0, 12)$, $P(8, -2, 8)$ и
$W(15, -3, -1)$.

Найти угол между прямой $BP$ и плоскостью $z = 0$, угол между $BK$ и $KW$, угол между плоскостями $BPK$ 
и $PKW$.

Составить уравнение: 

a) плоскости, параллельной $BPK$ и проходящей через $W$,

b) плоскости, перпендикулярной $BPK$ и проходящей через $B$ и $K$,

c) прямой, перпендикулярной $BPK$ и проходящей через $K$.

Вариант N94

Даны точки в пространстве
$D(22, -8, -14)$, $H(7, -5, -8)$, $M(7, -8, -8)$ и
$W(9, -1, -3)$.

Найти угол между прямой $MH$ и плоскостью $z = 0$, угол между $MW$ и $WD$, угол между плоскостями $MHW$ 
и $HWD$.

Составить уравнение: 

a) плоскости, параллельной $MHW$ и проходящей через $D$,

b) плоскости, перпендикулярной $MHW$ и проходящей через $M$ и $W$,

c) прямой, перпендикулярной $MHW$ и проходящей через $W$.

Вариант N95

Даны точки в пространстве
$H(-3, 2, 9)$, $K(-7, -1, 7)$, $V(-8, 2, -4)$ и
$W(-8, -4, 5)$.

Найти угол между прямой $WK$ и плоскостью $z = 0$, угол между $WH$ и $HV$, угол между плоскостями $WKH$ 
и $KHV$.

Составить уравнение: 

a) плоскости, параллельной $WKH$ и проходящей через $V$,

b) плоскости, перпендикулярной $WKH$ и проходящей через $W$ и $H$,

c) прямой, перпендикулярной $WKH$ и проходящей через $H$.

Вариант N96

Даны точки в пространстве
$A(6, -3, -4)$, $G(9, 3, 1)$, $H(16, -3, -10)$ и
$W(6, -1, -4)$.

Найти угол между прямой $AW$ и плоскостью $z = 0$, угол между $AG$ и $GH$, угол между плоскостями $AWG$ 
и $WGH$.

Составить уравнение: 

a) плоскости, параллельной $AWG$ и проходящей через $H$,

b) плоскости, перпендикулярной $AWG$ и проходящей через $A$ и $G$,

c) прямой, перпендикулярной $AWG$ и проходящей через $G$.

Вариант N97

Даны точки в пространстве
$A(5, -7, 5)$, $G(1, -1, 2)$, $P(4, 1, 2)$ и
$V(7, 4, 4)$.

Найти угол между прямой $GP$ и плоскостью $z = 0$, угол между $GV$ и $VA$, угол между плоскостями $GPV$ 
и $PVA$.

Составить уравнение: 

a) плоскости, параллельной $GPV$ и проходящей через $A$,

b) плоскости, перпендикулярной $GPV$ и проходящей через $G$ и $V$,

c) прямой, перпендикулярной $GPV$ и проходящей через $V$.

Вариант N98

Даны точки в пространстве
$B(-4, -5, 3)$, $K(-7, -8, 2)$, $M(0, 0, 7)$ и
$V(0, -16, 5)$.

Найти угол между прямой $KB$ и плоскостью $z = 0$, угол между $KM$ и $MV$, угол между плоскостями $KBM$ 
и $BMV$.

Составить уравнение: 

a) плоскости, параллельной $KBM$ и проходящей через $V$,

b) плоскости, перпендикулярной $KBM$ и проходящей через $K$ и $M$,

c) прямой, перпендикулярной $KBM$ и проходящей через $M$.

Вариант N99

Даны точки в пространстве
$C(-5, 3, -4)$, $D(-6, -2, 5)$, $F(0, 11, 1)$ и
$G(-2, 6, -2)$.

Найти угол между прямой $CG$ и плоскостью $z = 0$, угол между $CF$ и $FD$, угол между плоскостями $CGF$ 
и $GFD$.

Составить уравнение: 

a) плоскости, параллельной $CGF$ и проходящей через $D$,

b) плоскости, перпендикулярной $CGF$ и проходящей через $C$ и $F$,

c) прямой, перпендикулярной $CGF$ и проходящей через $F$.

Вариант N100

Даны точки в пространстве
$A(1, 5, 12)$, $B(4, 11, 8)$, $D(3, 9, 7)$ и
$M(0, 8, 7)$.

Найти угол между прямой $MD$ и плоскостью $z = 0$, угол между $MB$ и $BA$, угол между плоскостями $MDB$ 
и $DBA$.

Составить уравнение: 

a) плоскости, параллельной $MDB$ и проходящей через $A$,

b) плоскости, перпендикулярной $MDB$ и проходящей через $M$ и $B$,

c) прямой, перпендикулярной $MDB$ и проходящей через $B$.

Вариант N101

Даны точки в пространстве
$D(3, -2, 1)$, $G(5, -2, 4)$, $Q(-3, -3, 5)$ и
$U(6, 0, 6)$.

Найти угол между прямой $DG$ и плоскостью $z = 0$, угол между $DU$ и $UQ$, угол между плоскостями $DGU$ 
и $GUQ$.

Составить уравнение: 

a) плоскости, параллельной $DGU$ и проходящей через $Q$,

b) плоскости, перпендикулярной $DGU$ и проходящей через $D$ и $U$,

c) прямой, перпендикулярной $DGU$ и проходящей через $U$.

Вариант N102

Даны точки в пространстве
$A(8, 3, -2)$, $C(6, 1, -5)$, $F(6, 5, -12)$ и
$V(6, -1, -8)$.

Найти угол между прямой $VC$ и плоскостью $z = 0$, угол между $VA$ и $AF$, угол между плоскостями $VCA$ 
и $CAF$.

Составить уравнение: 

a) плоскости, параллельной $VCA$ и проходящей через $F$,

b) плоскости, перпендикулярной $VCA$ и проходящей через $V$ и $A$,

c) прямой, перпендикулярной $VCA$ и проходящей через $A$.

Вариант N103

Даны точки в пространстве
$C(13, -11, 1)$, $D(13, 9, -7)$, $H(9, 7, -9)$ и
$M(14, 12, -2)$.

Найти угол между прямой $HD$ и плоскостью $z = 0$, угол между $HM$ и $MC$, угол между плоскостями $HDM$ 
и $DMC$.

Составить уравнение: 

a) плоскости, параллельной $HDM$ и проходящей через $C$,

b) плоскости, перпендикулярной $HDM$ и проходящей через $H$ и $M$,

c) прямой, перпендикулярной $HDM$ и проходящей через $M$.

Вариант N104

Даны точки в пространстве
$D(7, -3, 13)$, $P(16, -4, 0)$, $R(5, -5, 8)$ и
$U(5, -8, 6)$.

Найти угол между прямой $UR$ и плоскостью $z = 0$, угол между $UD$ и $DP$, угол между плоскостями $URD$ 
и $RDP$.

Составить уравнение: 

a) плоскости, параллельной $URD$ и проходящей через $P$,

b) плоскости, перпендикулярной $URD$ и проходящей через $U$ и $D$,

c) прямой, перпендикулярной $URD$ и проходящей через $D$.

Вариант N105

Даны точки в пространстве
$F(13, 6, -6)$, $Q(9, 6, -9)$, $U(15, 8, -1)$ и
$V(3, -8, -1)$.

Найти угол между прямой $QF$ и плоскостью $z = 0$, угол между $QU$ и $UV$, угол между плоскостями $QFU$ 
и $FUV$.

Составить уравнение: 

a) плоскости, параллельной $QFU$ и проходящей через $V$,

b) плоскости, перпендикулярной $QFU$ и проходящей через $Q$ и $U$,

c) прямой, перпендикулярной $QFU$ и проходящей через $U$.

Вариант N106

Даны точки в пространстве
$C(-2, -6, -8)$, $K(-2, -4, -8)$, $P(2, -6, -12)$ и
$V(0, -3, -6)$.

Найти угол между прямой $CK$ и плоскостью $z = 0$, угол между $CV$ и $VP$, угол между плоскостями $CKV$ 
и $KVP$.

Составить уравнение: 

a) плоскости, параллельной $CKV$ и проходящей через $P$,

b) плоскости, перпендикулярной $CKV$ и проходящей через $C$ и $V$,

c) прямой, перпендикулярной $CKV$ и проходящей через $V$.

Вариант N107

Даны точки в пространстве
$C(9, -5, 0)$, $F(1, 13, -7)$, $R(14, -3, 1)$ и
$V(8, -6, -4)$.

Найти угол между прямой $VC$ и плоскостью $z = 0$, угол между $VR$ и $RF$, угол между плоскостями $VCR$ 
и $CRF$.

Составить уравнение: 

a) плоскости, параллельной $VCR$ и проходящей через $F$,

b) плоскости, перпендикулярной $VCR$ и проходящей через $V$ и $R$,

c) прямой, перпендикулярной $VCR$ и проходящей через $R$.

Вариант N108

Даны точки в пространстве
$K(-2, -4, 1)$, $P(2, -1, 6)$, $Q(8, -14, -1)$ и
$W(0, -2, 1)$.

Найти угол между прямой $KW$ и плоскостью $z = 0$, угол между $KP$ и $PQ$, угол между плоскостями $KWP$ 
и $WPQ$.

Составить уравнение: 

a) плоскости, параллельной $KWP$ и проходящей через $Q$,

b) плоскости, перпендикулярной $KWP$ и проходящей через $K$ и $P$,

c) прямой, перпендикулярной $KWP$ и проходящей через $P$.

Вариант N109

Даны точки в пространстве
$B(8, -9, -8)$, $D(14, -7, -4)$, $H(12, -25, -6)$ и
$P(12, -8, -8)$.

Найти угол между прямой $BP$ и плоскостью $z = 0$, угол между $BD$ и $DH$, угол между плоскостями $BPD$ 
и $PDH$.

Составить уравнение: 

a) плоскости, параллельной $BPD$ и проходящей через $H$,

b) плоскости, перпендикулярной $BPD$ и проходящей через $B$ и $D$,

c) прямой, перпендикулярной $BPD$ и проходящей через $D$.

Вариант N110

Даны точки в пространстве
$G(2, -6, 3)$, $P(-4, -5, 9)$, $V(-2, -4, 14)$ и
$W(-5, -7, 6)$.

Найти угол между прямой $WP$ и плоскостью $z = 0$, угол между $WV$ и $VG$, угол между плоскостями $WPV$ 
и $PVG$.

Составить уравнение: 

a) плоскости, параллельной $WPV$ и проходящей через $G$,

b) плоскости, перпендикулярной $WPV$ и проходящей через $W$ и $V$,

c) прямой, перпендикулярной $WPV$ и проходящей через $V$.

Вариант N111

Даны точки в пространстве
$C(-6, -5, 9)$, $K(-7, -8, 5)$, $M(-7, -7, 6)$ и
$V(-6, -7, 4)$.

Найти угол между прямой $KM$ и плоскостью $z = 0$, угол между $KC$ и $CV$, угол между плоскостями $KMC$ 
и $MCV$.

Составить уравнение: 

a) плоскости, параллельной $KMC$ и проходящей через $V$,

b) плоскости, перпендикулярной $KMC$ и проходящей через $K$ и $C$,

c) прямой, перпендикулярной $KMC$ и проходящей через $C$.

Вариант N112

Даны точки в пространстве
$H(8, 10, 12)$, $P(2, 6, 7)$, $R(5, 14, -3)$ и
$V(4, 9, 10)$.

Найти угол между прямой $PV$ и плоскостью $z = 0$, угол между $PH$ и $HR$, угол между плоскостями $PVH$ 
и $VHR$.

Составить уравнение: 

a) плоскости, параллельной $PVH$ и проходящей через $R$,

b) плоскости, перпендикулярной $PVH$ и проходящей через $P$ и $H$,

c) прямой, перпендикулярной $PVH$ и проходящей через $H$.

Вариант N113

Даны точки в пространстве
$F(9, 1, 5)$, $H(8, -1, 4)$, $K(8, -3, 3)$ и
$M(8, -2, 1)$.

Найти угол между прямой $KH$ и плоскостью $z = 0$, угол между $KF$ и $FM$, угол между плоскостями $KHF$ 
и $HFM$.

Составить уравнение: 

a) плоскости, параллельной $KHF$ и проходящей через $M$,

b) плоскости, перпендикулярной $KHF$ и проходящей через $K$ и $F$,

c) прямой, перпендикулярной $KHF$ и проходящей через $F$.

Вариант N114

Даны точки в пространстве
$A(7, -5, -9)$, $B(9, -5, -7)$, $C(-1, -11, -1)$ и
$U(10, -1, -3)$.

Найти угол между прямой $AB$ и плоскостью $z = 0$, угол между $AU$ и $UC$, угол между плоскостями $ABU$ 
и $BUC$.

Составить уравнение: 

a) плоскости, параллельной $ABU$ и проходящей через $C$,

b) плоскости, перпендикулярной $ABU$ и проходящей через $A$ и $U$,

c) прямой, перпендикулярной $ABU$ и проходящей через $U$.

Вариант N115

Даны точки в пространстве
$P(-1, 3, 4)$, $U(-5, 2, 4)$, $V(1, 7, 9)$ и
$W(0, -18, 18)$.

Найти угол между прямой $UP$ и плоскостью $z = 0$, угол между $UV$ и $VW$, угол между плоскостями $UPV$ 
и $PVW$.

Составить уравнение: 

a) плоскости, параллельной $UPV$ и проходящей через $W$,

b) плоскости, перпендикулярной $UPV$ и проходящей через $U$ и $V$,

c) прямой, перпендикулярной $UPV$ и проходящей через $V$.

Вариант N116

Даны точки в пространстве
$K(7, -1, 2)$, $M(19, -5, -2)$, $P(9, 2, 5)$ и
$Q(11, 3, 10)$.

Найти угол между прямой $KP$ и плоскостью $z = 0$, угол между $KQ$ и $QM$, угол между плоскостями $KPQ$ 
и $PQM$.

Составить уравнение: 

a) плоскости, параллельной $KPQ$ и проходящей через $M$,

b) плоскости, перпендикулярной $KPQ$ и проходящей через $K$ и $Q$,

c) прямой, перпендикулярной $KPQ$ и проходящей через $Q$.

Вариант N117

Даны точки в пространстве
$B(3, -8, 7)$, $C(3, -12, 11)$, $U(5, -6, 9)$ и
$W(6, -3, 12)$.

Найти угол между прямой $BU$ и плоскостью $z = 0$, угол между $BW$ и $WC$, угол между плоскостями $BUW$ 
и $UWC$.

Составить уравнение: 

a) плоскости, параллельной $BUW$ и проходящей через $C$,

b) плоскости, перпендикулярной $BUW$ и проходящей через $B$ и $W$,

c) прямой, перпендикулярной $BUW$ и проходящей через $W$.

Вариант N118

Даны точки в пространстве
$B(-7, -8, -3)$, $Q(-6, -6, 1)$, $R(-13, -13, -4)$ и
$U(-9, -9, -7)$.

Найти угол между прямой $UB$ и плоскостью $z = 0$, угол между $UQ$ и $QR$, угол между плоскостями $UBQ$ 
и $BQR$.

Составить уравнение: 

a) плоскости, параллельной $UBQ$ и проходящей через $R$,

b) плоскости, перпендикулярной $UBQ$ и проходящей через $U$ и $Q$,

c) прямой, перпендикулярной $UBQ$ и проходящей через $Q$.

Вариант N119

Даны точки в пространстве
$D(-2, 4, 0)$, $H(-4, 3, -1)$, $K(-8, 2, -6)$ и
$M(-8, 0, -4)$.

Найти угол между прямой $MH$ и плоскостью $z = 0$, угол между $MD$ и $DK$, угол между плоскостями $MHD$ 
и $HDK$.

Составить уравнение: 

a) плоскости, параллельной $MHD$ и проходящей через $K$,

b) плоскости, перпендикулярной $MHD$ и проходящей через $M$ и $D$,

c) прямой, перпендикулярной $MHD$ и проходящей через $D$.

Вариант N120

Даны точки в пространстве
$A(-3, -2, -5)$, $G(-3, -3, -6)$, $H(-2, 0, -3)$ и
$K(-3, -2, -7)$.

Найти угол между прямой $GA$ и плоскостью $z = 0$, угол между $GH$ и $HK$, угол между плоскостями $GAH$ 
и $AHK$.

Составить уравнение: 

a) плоскости, параллельной $GAH$ и проходящей через $K$,

b) плоскости, перпендикулярной $GAH$ и проходящей через $G$ и $H$,

c) прямой, перпендикулярной $GAH$ и проходящей через $H$.

Вариант N121

Даны точки в пространстве
$A(-5, 6, -8)$, $B(-2, 9, -7)$, $D(-7, 4, -2)$ и
$F(-7, 6, -8)$.

Найти угол между прямой $FA$ и плоскостью $z = 0$, угол между $FB$ и $BD$, угол между плоскостями $FAB$ 
и $ABD$.

Составить уравнение: 

a) плоскости, параллельной $FAB$ и проходящей через $D$,

b) плоскости, перпендикулярной $FAB$ и проходящей через $F$ и $B$,

c) прямой, перпендикулярной $FAB$ и проходящей через $B$.

Вариант N122

Даны точки в пространстве
$B(5, -8, -3)$, $D(10, 1, 0)$, $K(6, -4, -3)$ и
$W(17, -11, -14)$.

Найти угол между прямой $BK$ и плоскостью $z = 0$, угол между $BD$ и $DW$, угол между плоскостями $BKD$ 
и $KDW$.

Составить уравнение: 

a) плоскости, параллельной $BKD$ и проходящей через $W$,

b) плоскости, перпендикулярной $BKD$ и проходящей через $B$ и $D$,

c) прямой, перпендикулярной $BKD$ и проходящей через $D$.

Вариант N123

Даны точки в пространстве
$F(6, 7, 1)$, $H(9, -2, -2)$, $M(3, 4, 1)$ и
$P(11, 11, 3)$.

Найти угол между прямой $MF$ и плоскостью $z = 0$, угол между $MP$ и $PH$, угол между плоскостями $MFP$ 
и $FPH$.

Составить уравнение: 

a) плоскости, параллельной $MFP$ и проходящей через $H$,

b) плоскости, перпендикулярной $MFP$ и проходящей через $M$ и $P$,

c) прямой, перпендикулярной $MFP$ и проходящей через $P$.

Вариант N124

Даны точки в пространстве
$D(-1, -2, -3)$, $G(-7, 2, -11)$, $M(-1, -4, -5)$ и
$P(2, 2, -2)$.

Найти угол между прямой $MD$ и плоскостью $z = 0$, угол между $MP$ и $PG$, угол между плоскостями $MDP$ 
и $DPG$.

Составить уравнение: 

a) плоскости, параллельной $MDP$ и проходящей через $G$,

b) плоскости, перпендикулярной $MDP$ и проходящей через $M$ и $P$,

c) прямой, перпендикулярной $MDP$ и проходящей через $P$.

Вариант N125

Даны точки в пространстве
$A(3, -1, 8)$, $D(8, -12, 11)$, $F(1, -3, 7)$ и
$H(4, 2, 13)$.

Найти угол между прямой $FA$ и плоскостью $z = 0$, угол между $FH$ и $HD$, угол между плоскостями $FAH$ 
и $AHD$.

Составить уравнение: 

a) плоскости, параллельной $FAH$ и проходящей через $D$,

b) плоскости, перпендикулярной $FAH$ и проходящей через $F$ и $H$,

c) прямой, перпендикулярной $FAH$ и проходящей через $H$.

Вариант N126

Даны точки в пространстве
$B(0, 10, 6)$, $H(-9, 9, 3)$, $M(-4, 5, 4)$ и
$V(-4, 5, 3)$.

Найти угол между прямой $VM$ и плоскостью $z = 0$, угол между $VB$ и $BH$, угол между плоскостями $VMB$ 
и $MBH$.

Составить уравнение: 

a) плоскости, параллельной $VMB$ и проходящей через $H$,

b) плоскости, перпендикулярной $VMB$ и проходящей через $V$ и $B$,

c) прямой, перпендикулярной $VMB$ и проходящей через $B$.

Вариант N127

Даны точки в пространстве
$A(5, 16, 2)$, $D(1, 9, -1)$, $G(1, 12, -1)$ и
$H(10, 9, -13)$.

Найти угол между прямой $DG$ и плоскостью $z = 0$, угол между $DA$ и $AH$, угол между плоскостями $DGA$ 
и $GAH$.

Составить уравнение: 

a) плоскости, параллельной $DGA$ и проходящей через $H$,

b) плоскости, перпендикулярной $DGA$ и проходящей через $D$ и $A$,

c) прямой, перпендикулярной $DGA$ и проходящей через $A$.

Вариант N128

Даны точки в пространстве
$A(-15, 8, -9)$, $G(1, 2, 5)$, $P(-7, -6, -1)$ и
$Q(-4, -2, 3)$.

Найти угол между прямой $PQ$ и плоскостью $z = 0$, угол между $PG$ и $GA$, угол между плоскостями $PQG$ 
и $QGA$.

Составить уравнение: 

a) плоскости, параллельной $PQG$ и проходящей через $A$,

b) плоскости, перпендикулярной $PQG$ и проходящей через $P$ и $G$,

c) прямой, перпендикулярной $PQG$ и проходящей через $G$.

Вариант N129

Даны точки в пространстве
$A(-4, 7, 8)$, $B(-2, 11, 8)$, $P(1, 12, 11)$ и
$V(8, 1, -2)$.

Найти угол между прямой $AB$ и плоскостью $z = 0$, угол между $AP$ и $PV$, угол между плоскостями $ABP$ 
и $BPV$.

Составить уравнение: 

a) плоскости, параллельной $ABP$ и проходящей через $V$,

b) плоскости, перпендикулярной $ABP$ и проходящей через $A$ и $P$,

c) прямой, перпендикулярной $ABP$ и проходящей через $P$.

Вариант N130

Даны точки в пространстве
$A(2, 14, -3)$, $F(-1, 8, 4)$, $P(2, 10, 7)$ и
$W(7, 11, 10)$.

Найти угол между прямой $FP$ и плоскостью $z = 0$, угол между $FW$ и $WA$, угол между плоскостями $FPW$ 
и $PWA$.

Составить уравнение: 

a) плоскости, параллельной $FPW$ и проходящей через $A$,

b) плоскости, перпендикулярной $FPW$ и проходящей через $F$ и $W$,

c) прямой, перпендикулярной $FPW$ и проходящей через $W$.

Вариант N131

Даны точки в пространстве
$B(1, 7, -9)$, $D(2, 8, -5)$, $F(-16, 16, -7)$ и
$K(5, 13, -2)$.

Найти угол между прямой $BD$ и плоскостью $z = 0$, угол между $BK$ и $KF$, угол между плоскостями $BDK$ 
и $DKF$.

Составить уравнение: 

a) плоскости, параллельной $BDK$ и проходящей через $F$,

b) плоскости, перпендикулярной $BDK$ и проходящей через $B$ и $K$,

c) прямой, перпендикулярной $BDK$ и проходящей через $K$.

Вариант N132

Даны точки в пространстве
$A(3, -7, 11)$, $C(-8, -3, 11)$, $V(-2, -9, 9)$ и
$W(-4, -9, 7)$.

Найти угол между прямой $WV$ и плоскостью $z = 0$, угол между $WA$ и $AC$, угол между плоскостями $WVA$ 
и $VAC$.

Составить уравнение: 

a) плоскости, параллельной $WVA$ и проходящей через $C$,

b) плоскости, перпендикулярной $WVA$ и проходящей через $W$ и $A$,

c) прямой, перпендикулярной $WVA$ и проходящей через $A$.

Вариант N133

Даны точки в пространстве
$F(-3, 12, 2)$, $K(-9, 8, -4)$, $M(-15, 20, -6)$ и
$V(-8, 9, -1)$.

Найти угол между прямой $KV$ и плоскостью $z = 0$, угол между $KF$ и $FM$, угол между плоскостями $KVF$ 
и $VFM$.

Составить уравнение: 

a) плоскости, параллельной $KVF$ и проходящей через $M$,

b) плоскости, перпендикулярной $KVF$ и проходящей через $K$ и $F$,

c) прямой, перпендикулярной $KVF$ и проходящей через $F$.

Вариант N134

Даны точки в пространстве
$B(9, 5, 5)$, $P(1, 0, -3)$, $Q(4, 1, 0)$ и
$U(-6, 0, 4)$.

Найти угол между прямой $PQ$ и плоскостью $z = 0$, угол между $PB$ и $BU$, угол между плоскостями $PQB$ 
и $QBU$.

Составить уравнение: 

a) плоскости, параллельной $PQB$ и проходящей через $U$,

b) плоскости, перпендикулярной $PQB$ и проходящей через $P$ и $B$,

c) прямой, перпендикулярной $PQB$ и проходящей через $B$.

Вариант N135

Даны точки в пространстве
$D(-1, 14, 9)$, $F(-5, 9, 4)$, $H(-10, 13, 3)$ и
$M(-5, 9, 3)$.

Найти угол между прямой $MF$ и плоскостью $z = 0$, угол между $MD$ и $DH$, угол между плоскостями $MFD$ 
и $FDH$.

Составить уравнение: 

a) плоскости, параллельной $MFD$ и проходящей через $H$,

b) плоскости, перпендикулярной $MFD$ и проходящей через $M$ и $D$,

c) прямой, перпендикулярной $MFD$ и проходящей через $D$.

Вариант N136

Даны точки в пространстве
$A(-8, -5, -8)$, $P(1, 3, -2)$, $Q(-6, -11, -3)$ и
$R(-4, -2, -6)$.

Найти угол между прямой $AR$ и плоскостью $z = 0$, угол между $AP$ и $PQ$, угол между плоскостями $ARP$ 
и $RPQ$.

Составить уравнение: 

a) плоскости, параллельной $ARP$ и проходящей через $Q$,

b) плоскости, перпендикулярной $ARP$ и проходящей через $A$ и $P$,

c) прямой, перпендикулярной $ARP$ и проходящей через $P$.

Вариант N137

Даны точки в пространстве
$D(-1, -5, 4)$, $F(3, -1, 2)$, $R(1, -3, 0)$ и
$V(5, 3, 5)$.

Найти угол между прямой $RF$ и плоскостью $z = 0$, угол между $RV$ и $VD$, угол между плоскостями $RFV$ 
и $FVD$.

Составить уравнение: 

a) плоскости, параллельной $RFV$ и проходящей через $D$,

b) плоскости, перпендикулярной $RFV$ и проходящей через $R$ и $V$,

c) прямой, перпендикулярной $RFV$ и проходящей через $V$.

Вариант N138

Даны точки в пространстве
$B(-5, -1, -1)$, $D(-20, -4, 8)$, $H(-1, 3, 4)$ и
$U(-8, -1, -4)$.

Найти угол между прямой $UB$ и плоскостью $z = 0$, угол между $UH$ и $HD$, угол между плоскостями $UBH$ 
и $BHD$.

Составить уравнение: 

a) плоскости, параллельной $UBH$ и проходящей через $D$,

b) плоскости, перпендикулярной $UBH$ и проходящей через $U$ и $H$,

c) прямой, перпендикулярной $UBH$ и проходящей через $H$.

Вариант N139

Даны точки в пространстве
$K(-9, 9, 7)$, $M(6, 10, 9)$, $P(3, 6, 3)$ и
$V(4, 6, 6)$.

Найти угол между прямой $PV$ и плоскостью $z = 0$, угол между $PM$ и $MK$, угол между плоскостями $PVM$ 
и $VMK$.

Составить уравнение: 

a) плоскости, параллельной $PVM$ и проходящей через $K$,

b) плоскости, перпендикулярной $PVM$ и проходящей через $P$ и $M$,

c) прямой, перпендикулярной $PVM$ и проходящей через $M$.

Вариант N140

Даны точки в пространстве
$G(1, -1, 10)$, $M(12, 5, 6)$, $P(9, 1, 2)$ и
$R(13, 9, 8)$.

Найти угол между прямой $PM$ и плоскостью $z = 0$, угол между $PR$ и $RG$, угол между плоскостями $PMR$ 
и $MRG$.

Составить уравнение: 

a) плоскости, параллельной $PMR$ и проходящей через $G$,

b) плоскости, перпендикулярной $PMR$ и проходящей через $P$ и $R$,

c) прямой, перпендикулярной $PMR$ и проходящей через $R$.

Вариант N141

Даны точки в пространстве
$C(12, -10, -12)$, $F(-4, 6, -4)$, $G(0, 10, -4)$ и
$Q(5, 13, 0)$.

Найти угол между прямой $FG$ и плоскостью $z = 0$, угол между $FQ$ и $QC$, угол между плоскостями $FGQ$ 
и $GQC$.

Составить уравнение: 

a) плоскости, параллельной $FGQ$ и проходящей через $C$,

b) плоскости, перпендикулярной $FGQ$ и проходящей через $F$ и $Q$,

c) прямой, перпендикулярной $FGQ$ и проходящей через $Q$.

Вариант N142

Даны точки в пространстве
$D(1, -4, 2)$, $K(6, -5, 2)$, $Q(7, 0, 4)$ и
$U(6, -5, 3)$.

Найти угол между прямой $KU$ и плоскостью $z = 0$, угол между $KQ$ и $QD$, угол между плоскостями $KUQ$ 
и $UQD$.

Составить уравнение: 

a) плоскости, параллельной $KUQ$ и проходящей через $D$,

b) плоскости, перпендикулярной $KUQ$ и проходящей через $K$ и $Q$,

c) прямой, перпендикулярной $KUQ$ и проходящей через $Q$.

Вариант N143

Даны точки в пространстве
$D(16, 0, 12)$, $F(11, -4, 10)$, $K(-1, 4, 2)$ и
$Q(7, -8, 6)$.

Найти угол между прямой $QF$ и плоскостью $z = 0$, угол между $QD$ и $DK$, угол между плоскостями $QFD$ 
и $FDK$.

Составить уравнение: 

a) плоскости, параллельной $QFD$ и проходящей через $K$,

b) плоскости, перпендикулярной $QFD$ и проходящей через $Q$ и $D$,

c) прямой, перпендикулярной $QFD$ и проходящей через $D$.

Вариант N144

Даны точки в пространстве
$B(-1, 15, -6)$, $F(-6, 6, -9)$, $M(-3, 10, -7)$ и
$R(-12, 7, -2)$.

Найти угол между прямой $FM$ и плоскостью $z = 0$, угол между $FB$ и $BR$, угол между плоскостями $FMB$ 
и $MBR$.

Составить уравнение: 

a) плоскости, параллельной $FMB$ и проходящей через $R$,

b) плоскости, перпендикулярной $FMB$ и проходящей через $F$ и $B$,

c) прямой, перпендикулярной $FMB$ и проходящей через $B$.

Вариант N145

Даны точки в пространстве
$A(1, 6, 2)$, $P(-8, 15, -15)$, $Q(-4, 0, -5)$ и
$U(-4, 2, -2)$.

Найти угол между прямой $QU$ и плоскостью $z = 0$, угол между $QA$ и $AP$, угол между плоскостями $QUA$ 
и $UAP$.

Составить уравнение: 

a) плоскости, параллельной $QUA$ и проходящей через $P$,

b) плоскости, перпендикулярной $QUA$ и проходящей через $Q$ и $A$,

c) прямой, перпендикулярной $QUA$ и проходящей через $A$.

Вариант N146

Даны точки в пространстве
$B(-4, 7, 4)$, $H(-8, 7, 0)$, $R(-8, 5, 2)$ и
$V(-9, 5, 2)$.

Найти угол между прямой $RV$ и плоскостью $z = 0$, угол между $RB$ и $BH$, угол между плоскостями $RVB$ 
и $VBH$.

Составить уравнение: 

a) плоскости, параллельной $RVB$ и проходящей через $H$,

b) плоскости, перпендикулярной $RVB$ и проходящей через $R$ и $B$,

c) прямой, перпендикулярной $RVB$ и проходящей через $B$.

Вариант N147

Даны точки в пространстве
$A(2, 10, -14)$, $D(-5, 7, 3)$, $K(-9, 3, -6)$ и
$V(-8, 6, -2)$.

Найти угол между прямой $KV$ и плоскостью $z = 0$, угол между $KD$ и $DA$, угол между плоскостями $KVD$ 
и $VDA$.

Составить уравнение: 

a) плоскости, параллельной $KVD$ и проходящей через $A$,

b) плоскости, перпендикулярной $KVD$ и проходящей через $K$ и $D$,

c) прямой, перпендикулярной $KVD$ и проходящей через $D$.

Вариант N148

Даны точки в пространстве
$D(7, 2, 3)$, $F(-2, 11, 3)$, $K(8, 3, 6)$ и
$R(12, 7, 9)$.

Найти угол между прямой $DK$ и плоскостью $z = 0$, угол между $DR$ и $RF$, угол между плоскостями $DKR$ 
и $KRF$.

Составить уравнение: 

a) плоскости, параллельной $DKR$ и проходящей через $F$,

b) плоскости, перпендикулярной $DKR$ и проходящей через $D$ и $R$,

c) прямой, перпендикулярной $DKR$ и проходящей через $R$.

Вариант N149

Даны точки в пространстве
$A(11, 9, -8)$, $D(7, 7, -8)$, $Q(14, 10, -7)$ и
$V(7, 8, -9)$.

Найти угол между прямой $VA$ и плоскостью $z = 0$, угол между $VQ$ и $QD$, угол между плоскостями $VAQ$ 
и $AQD$.

Составить уравнение: 

a) плоскости, параллельной $VAQ$ и проходящей через $D$,

b) плоскости, перпендикулярной $VAQ$ и проходящей через $V$ и $Q$,

c) прямой, перпендикулярной $VAQ$ и проходящей через $Q$.

Вариант N150

Даны точки в пространстве
$A(-10, -5, 1)$, $H(-1, -7, 8)$, $R(-2, -9, 1)$ и
$U(-2, -9, 5)$.

Найти угол между прямой $RU$ и плоскостью $z = 0$, угол между $RH$ и $HA$, угол между плоскостями $RUH$ 
и $UHA$.

Составить уравнение: 

a) плоскости, параллельной $RUH$ и проходящей через $A$,

b) плоскости, перпендикулярной $RUH$ и проходящей через $R$ и $H$,

c) прямой, перпендикулярной $RUH$ и проходящей через $H$.

\end{document}